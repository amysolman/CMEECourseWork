\documentclass[11pt, a4paper, twoside]{report}

\usepackage[british]{babel}
\usepackage[useregional]{datetime2}
\DTMlangsetup[en-GB]{showdayofmonth=false} %to remove day from date
\usepackage[utf8x]{inputenc}
\usepackage[T1]{fontenc}
\usepackage{listings}
\usepackage{hyperref}
\hypersetup{colorlinks=false}
\usepackage{lscape} %to put things landscape
\usepackage{rotating} %to put things landscape, again
\usepackage{pdflscape} %trying landscape again
%\usepackage{subfigure} %removed as it was interfering with subfig package
\usepackage{amsmath} %for some maths stuff! 
\usepackage{graphicx}
\usepackage[colorinlistoftodos]{todonotes}
\usepackage{csvsimple}
\usepackage{lscape}%for landscape table in sup mat
\usepackage{apalike} %for round brackets in references
\usepackage{titlesec} %for hiding chapter title
\usepackage{subfig} % for getting 4 figures looking nice
\usepackage{setspace} %for spacing
\usepackage{listings} %for my R code
\usepackage{tikz} %for doing little circles in line with text
%\onehalfspacing %one and a half line spacing
\setstretch{1.5}
\titleformat{\chapter}[display]
  {\normalfont\bfseries}{}{0pt}{\Huge}

%%%%%This stuff is for generating tables%%%%
\usepackage{booktabs} % For \toprule, \midrule and \bottomrule
\usepackage{siunitx} % Formats the units and values
\usepackage{pgfplotstable} % Generates table from .csv

% Setup siunitx:
\sisetup{
  round-mode          = places, % Rounds numbers
  round-precision     = 3, % to 2 places
}
%%%%%%%%%%%%%%%%%%%%%%%%%%%

%% Sets page size and margins
\usepackage[a4paper,top=3cm,bottom=2cm,left=3cm,right=3cm,marginparwidth=1.75cm]{geometry}

\title{Is everything everywhere? Deterministic and stochastic processes in microbial biogeography}
\author{Amy Solman}
% Update supervisor and other title stuff in title/title.tex

\begin{document}
\begin{titlepage}

\newcommand{\HRule}{\rule{\linewidth}{0.5mm}} % Defines a new command for the horizontal lines, change thickness here

%----------------------------------------------------------------------------------------
%	LOGO SECTION
%----------------------------------------------------------------------------------------

\includegraphics[width=8cm]{logo.eps}\\[1cm] % Include a department/university logo - this will require the graphicx package
 
%----------------------------------------------------------------------------------------

\center % Center everything on the page

%----------------------------------------------------------------------------------------
%	HEADING SECTIONS
%----------------------------------------------------------------------------------------
%\quad\\[1.5cm]
%\textsc{\LARGE MSc Thesis}\\[1.5cm] % Name of your university/college
\textsc{\Large Imperial College London}\\[0.5cm] % Major heading such as course name
\textsc{\large Department of Life Sciences}\\[0.5cm] % Minor heading such as course title

%----------------------------------------------------------------------------------------
%	TITLE SECTION
%----------------------------------------------------------------------------------------
\makeatletter
\HRule \\[0.4cm]
{ \huge \bfseries \@title}\\[0.4cm] % Title of your document
{ \huge \bfseries \@author}\\[0.4cm]
\HRule \\[1.5cm]
 
%----------------------------------------------------------------------------------------
%	AUTHOR SECTION
%----------------------------------------------------------------------------------------

\begin{minipage}{0.4\textwidth}
\begin{flushleft} \large
\emph{External supervisor:}\\
Prof. Ryan Chisholm\\
University of Singapore \\
\end{flushleft}
\end{minipage}
~
\begin{minipage}{0.4\textwidth}
\begin{flushright} \large
\emph{Internal supervisors:} \\
Dr. James Rosindell\\
Imperial College London\\

Prof. Thomas Bell\\
Imperial College London
% Uncomment the following lines if there's a co-supervisor
%\\[1.2em] % Supervisor's Name
%\emph{Co-Supervisor:} \\
%Dr. Adam Smith % second marker's name
\end{flushright}
\end{minipage}\\[2cm]
\makeatother


%----------------------------------------------------------------------------------------
%	DATE SECTION
%----------------------------------------------------------------------------------------
{\large \today}\\[1cm] % Date, change the \today to a set date if you want to be precise
{\large A thesis submitted in partial fulfilment of the requirements for the degree of}\\%[0.5cm]
{\large {Master of Research at Imperial College London}}\\[0.5cm]

{\large Formatted in the journal style of Methods in Ecology and Evolution}\\%[0.5cm]
{\large Submitted for the MRes in Computational Methods in Ecology and Evolution}\\%[0.5cm]


\vfill % Fill the rest of the page with whitespace

\end{titlepage}

\renewcommand{\abstractname}{Declaration}
\begin{abstract}
\textbf{Concept:} The concept for this work came from Prof. Thomas Bell and Prof. Ryan Chisholm.\\
\noindent \textbf{Data:} All data for this analysis was collected by myself from the literature as listed in the Supplementary Materials section.\\
\noindent \textbf{Simulation:} All simulation code was written by myself, excluding function 2: coalescence\_test (simulation scripts) provided by Dr. James Rosindell.\\
\noindent \textbf{Model:} The Classic, Depth and Perimeter models, as well as critical area formulas, were supplied by Prof. Ryan Chisholm.\\
\noindent \textbf{Analysis:} I declare that all analysis was carried out by myself.\\
\noindent \textbf{Report:} I declare that the report was written by myself.\\
\\ \\ \\

\noindent \textbf{COVID-19} \\
\noindent The hypotheses presented in this report were originally investigated using a laboratory-based experiment in Professor Thomas Bell's Microbial Ecology Laboratory, Imperial College London, Silwood Park. Two months into the investigation, laboratory work was ceased due to the COVID-19 pandemic. The following report seeks to test the same hypotheses, using data from the literature.
\end{abstract}

\renewcommand{\abstractname}{Abstract}
\begin{abstract}
1. Microbial communities play an essential role in biogeochemical processes and ecosystem functioning. Despite their importance, relatively little is known about the mechanisms driving spatial scaling within these communities. \\

\noindent 2. The Theory of Island Biogeography predicts that species richness increases with island area through stochastic colonisation-extinction processes. It has been widely use to describe macroorganism spatial patterns, with little application to microbial communities.\\

\noindent 3. Despite the popularity of this theory, small islands often show no clear relationship between species richness and area. Chisholm \textit{et al.,} have addressed this by developing a unified theory of species-area relationships that transitions from a niche-structured regime at smaller spatial scales to stochastic mechanisms at larger spatial scales.\\

\noindent 4. This work modifies the Chisholm model to address microbial communities in a range of habitats. Through model fitting I assess to what extent microorganisms are subject to deterministic or stochastic processes. I also explore how the transition between regimes differs among habitat types and taxonomic groups. \\

\noindent 5. The models gave good fits to the data. Multiple regression analysis showed significant evidence that taxonomic group explained variance in the critical area of transition between regimes. Less motile groups exhibited higher critical areas supporting my hypothesis. Habitat type was non-significant in predicting critical area.\\

\noindent 6. A proportion of the datasets exhibited a biphasic species-area relationship. Broadly the critical area hypotheses of lower transition areas for more motile taxa was supported by the data. These results can assist in predicting the spatial scaling of microbial diversity, with application to climate change modelling. \\

\noindent \textbf{Keywords:} species-area relationships, taxa-area relationships, microbial biogeography, small-island effect, niche structure, colonisation-extinction balance, island biogeography
\end{abstract}

\tableofcontents
\listoffigures
\listoftables

\chapter{Introduction}

%What are SARs and why are they important?
%Why is the theory important? 
The species--area relationship (SAR) is one of the oldest fundamental ecological laws \cite{GooriahLeanaD2019Sedt}. SARs describe the relationship between community diversity and habitat area. The observation that species richness increases with sampling area, a positive SAR, has been observed for a broad range of faunal \cite{ricklefs1999roles} \cite{lomolino1982species} \cite{eadie1986lakes} and floral groups \cite{zacharias1990species} \cite{price2011phylogenetic}. The ubiquitous nature of positive SARs has been used to inform conservation practises in natural \cite{haila2002conceptual} \cite{samson1980island} and urban environments \cite{davis1978urban}. Whilst a large number of studies have examined macro-organism SARs, relatively little is known about the spatial scaling of microbial biodiversity. \\

%Why is it important?
{\texorpdfstring
\noindent UUnderstanding the factors that regulate microbial community structure is important as they play a vital role in biogeochemical cycling and ecosystem functioning \cite{griffiths2011bacterial}. Despite bacteria and fungi representing major contributors to soil biodiversity and processes, little is known about below-ground regulators of biodiversity \cite{griffiths2011bacterial} \cite{li2020island}. This is especially challenging as few terrestrial environments present insular habitats for microbial community dynamics to be easily studied. Microorganisms also play a metabolically active role in polar regions previously believed to be abiotic \cite{stibal2020glacial}. Rapid climate change is leading to the exposure of soils dominant in high-latitude carbon \cite{bradley2017microbial}. As these soils are colonised by microbial communities, biogeochemical transformations release CO\textsubscript{2}, CH\textsubscript{4} and N\textsubscript{2}O. Understanding the mechanisms that drive microbial colonisation of polar environments can help produce accurate models of greenhouse gas release \cite{malard2018microbial}.} \\

%History of Microbial SARs
\noindent Debate around the applicability of SARs to microbial systems stems from the assumption that they are limited only by niche-filtering, as articulated by Bass-Becking: `\textit{Everything is everywhere}, but, \textit{the environment selects}' \cite{baas1934geobiologie}. This classic tenet of microbiology assumes that the abundance, short generation times and small size of microorganisms gives them an almost cosmopolitan distribution \cite{GreenJessica2006Ssom}. High abundances increase the probability of transport between environments via an accidental vector. Small size also increases the likelihood of passive transport via air or water, leading to high dispersal rates \cite{GreenJessica2006Ssom}. Uninhibited dispersal may also be facilitated by dormancy as a biogeographical response \cite{LoceyKennethJ2010Stbw}. \\

%The species-area relationship power-law model
\noindent One of the most commonly used tools in biogeography is the power-law model:

\begin{equation}
S=cA^{z}
\end{equation}\\

\noindent Where \textit{S} is species richness as a function of area (\textit{A}), \textit{c} is a constant specific to that taxa/habitat and the \textit{z} exponent is the slope of the line associating area and species richness \cite{darcy2018island}. \textit{z} typically falls in the range of 0.1 to 0.3 for continuous habitats and 0.25 to 0.35 for insular habitats \cite{GreenJessica2006Ssom}. Microbial \textit{z} values are typically well below those seen in macro-organisms (z < 0.1), supporting the idea of cosmopolitan distribution \cite{GreenJessica2006Ssom}. \\

%Problems with Microbial SARs
%Distribution maps define the annual or seasonal spatial distributions of functional groups and life stages, for simulating spatial patterns of predator-prey interactions
\noindent One of the limitations for microbial biogeography has been in quantifying taxa, given that many cannot be accurately identified using morphological techniques \cite{GreenJessica2006Ssom}. Commonly, microbial biogeography is concerned with taxa-area relationships (TARs), rather than SARs as microbial diversity is quantified in operational taxonomic groups (OTUs). Deciduous leaves as 'island' habitats for aquatic fungi found morphospecies-based diversity increased with leaf area, but next generation sequencing did not. \cite{duarte2017taxa}. With recent advances in molecular approaches such as single-celled sequencing, the genomes of previously uncultivated bacterial taxa are filling in the phylogenetic tree providing a higher resolution picture of microbial community structure \cite{lasken2014recent}. Limited data on temporal and spatial microbial distributions has led to a lack of detailed distribution maps. Distribution maps allow us to estimate the true number of taxa in a given environment when total counts are not available, without which estimated \textit{z} values may be artificially low \cite{GreenJessica2006Ssom}.\\


\noindent The mechanisms driving island SARs have been of particular interest to ecologists since the 1800s \cite{macdonald2018theory}. Islands are considered important paradigms for fragmented habitats as well as larger geographic regions \cite{simberloff1974equilibrium}. Their insular nature allows for ecological processes and patterns to be investigated in a simplified and relatively closed system. \\

%The general theory of island biogeography
%What is the theory about?

\begin{figure}[htp]

\centering
\includegraphics[width=.5\textwidth]{ColonisationDynamicEquilibrium.pdf}\hfill

\caption{Colonisation-Extinction Dynamic Equilibrium}
\label{fig:figure1}

\end{figure}

\noindent MacArthur and Wilson's Theory of Island Biogeography \cite{MacArthurRobertH1967Ttoi} is one of the most widely accepted island SAR theories. Explains the maintenance of biodiversity on islands through the stochastic processes of colonisation and extinction. The rates of these processes are determined by island area and isolation from the mainland. Islands that are nearer to source populations will experience a higher rate of immigration. This in turn can produce a rescue effect leading to decreased extinction rates \cite{brown1977turnover}. Larger islands will receive more immigrants as species actively target larger habitats with more resources, or will be more likely to immigrate randomly due to island size. A larger population is also less susceptible to inbreeding depressions and random extinction \cite{macdonald2018theory}. This results in higher species richness at the point of balance between immigration and extinction rates (i.e. the colonisation-extinction dynamic equilibrium, Figure 1.1) for larger, less isolated islands. \\

%What other mechanisms can drive patterns of biogeography?
\noindent It has been suggested that the stochastic significance of area in predicting species richness has been overplayed, to the exclusion of deterministic mechanisms such as interspecific relationships, biotic and abiotic factors \cite{abbott1974numbers}. This is due to empirical evidence suggesting that smaller islands do not always follow the positive SAR pattern  \cite{triantis2006re} \cite{sfenthourakis2009habitat}. MacArthur and Wilson noted that archipelagos showed unusual SARs, with smaller island species-richness varying independently of size \cite{MacArthurRobertH1967Ttoi}. It appears when smaller habitats within a broad range of spatial scales are assessed, both deterministic and stochastic patterns can emerge \cite{lomolino2001towards}. This exception to MacArthur and Wilson's putative ecological law has been dubbed the small-island effect (SIE). \\

%Explainations of the SIE
\noindent Several hypotheses have been offered to explain the SIE. The 'subsidized island biogeography' hypothesis suggests that smaller islands have a greater edge to interior ratio, thus receive a greater amount of nutrients per unit area \cite{barrett2003small} \cite{anderson2001subsidized}. Secondly, extinction rates on islands may operate independently of area due to their environmental instability and high temporal turnover, where major episodic disturbances periodically wipe out colonising species \cite{MacArthurRobertH1967Ttoi}. Thirdly, the 'habitat hypothesis' suggests that diversity is limited on smaller islands, compared to larger islands \cite{triantis2008evolutionary}. However, the environmental instability and habitat hypotheses contradict empirical data that indicate small islands have unusually high numbers of species. The Habitat--Diversity Hypothesis addresses the SIE phenomena by stating that as observation area increases we encounter a greater range of habitats \cite{EdwardF.Connor1979TSaB}. Therefore, the theory predicts that species richness should increase with habitat diversity, which varies independently of area \cite{macdonald2018theory}.  \\

%Chisholm et al model and research
\begin{figure}[htp]

\centering
\includegraphics[width=.5\textwidth]{LowImIslands.png}\hfill

\caption{A graphical representation of a simulation (using the Classic Model, see Methods) of three islands of varying size, with the same number of niches (\textit{K}=4) and \textbf{low immigration rate} (\textit{m\textsubscript{0}} = 0.03). Each of the three main squares represents an island. Each smaller square represents an individual niche. Each unique colour patch within a niche represents a unique species. The smallest island has one species her niche, the medium size island has four individuals per niche and the largest island has nine individuals per niche. Species richness on the smallest island is \textbf{4}, on the medium island is \textbf{5} and the large island is \textbf{6}}
\label{fig:figure2}

\end{figure}

\begin{figure}[htp]

\centering
\includegraphics[width=.5\textwidth]{HighImIslands.png}\hfill


\caption{A graphical representation of a simulation (using the Classic Model, see Methods) of three islands of varying size, with the same number of niches (\textit{K}=4) and \textbf{high immigration rate} (\textit{m\textsubscript{0}} = 0.9). Each of the three main squares represents an island. Each smaller square represents an individual niche. Each unique colour patch within a niche represents a unique species. The smallest island has one species her niche, the medium size island has four individuals per niche and the largest island has nine individuals per niche. Species richness on the smallest island is \textbf{4}, on the medium island is \textbf{15} and the large island is \textbf{33}}
\label{fig:figure3}

\end{figure}

\noindent Chisholm \textit{et al.,} (2016) explain both deterministic and stochastic SARs in a unified theory. They posit that this pattern of species-richness is due to a transition from a niche-structured regime on smaller islands, to a colonisation-extinction regime on larger islands. The niche-structured regime is characteristic of deterministic theories like the Habitat--Diversity Hypothesis, where habitat structure and intra- and interspecific interactions determine species richness \cite{chase2011disentangling}. The colonisation-extinction regime is characteristic of stochastic mechanisms such as the Theory of Island Biogeography and ecological neutral theory, where richness is dictated by random colonisation and extinction events, as well as ecological drift \cite{hubbell2001unified}. Chisholm \textit{et al.,} hypothesise that species richness on all islands is maintained by these two mechanisms. They suggest that niche diversity increases slowly with area, whilst immigration rate increases quickly. Thus smaller islands are constrained by niche-structured regimes, until a critical area threshold where species richness is constrained by immigration. Figures 1.2 and 1.3 show the effect of immigration rate and area on species richness, where each 'island' is made up of four niches and the different coloured patches inside each niche represent a species unique to that niche. For islands with low immigration (Figure 1.2) or high immigration (Figure 1.3), small islands harbour the same number of species. Small islands where each niche can support only a small number of individuals will be constrained by those niches. Larger islands have less species at lower immigration rates and more species at higher immigration rates. They are less constrained by the number or size of their niches, and their species richness is dictated by random immigration and extinction events. \\

\noindent Chisholm \textit{et al.,} developed a parsimonious mechanistic model to test their hypotheses, and applied it to 100 archipelago datasets. Their results supported the prediction that critical area will be lower for species with higher motility and less isolated habitats. \\

%what do we know about the mechanisms driving microbial TARs? Deterministic niche regimes vs stochastic neutral regimes
\noindent Previous research indicates that microbial TARs may be controlled by either deterministic, environmental mechanisms, or stochastic, neutral processes \cite{StegenJamesC2012Sada}. Phylogenetic analysis of subsurface microbial communities showed related taxa utilised similar habitats, illustrating that environmental filtering determined community composition \cite{StegenJamesC2012Sada}. Niche filtering also had a greater influence in the most spatially and temporally varied environments \cite{StegenJamesC2012Sada}. The relative strength of these mechanisms has also been shown to vary with community functionality \cite{CarusoTancredi2011Sadp}. \\

\noindent Whilst both stochastic and deterministic processes are demonstrated for microbial communities, few studies discuss the transition of mechanisms across a spatial scale. An investigation of phytoplankton TARs in water bodies indicated that for the smallest spatial scales, niche relations determine OTU richness, before transitioning to an immigration dominated regime \cite{varbiro2017functional}. The SIE has also been seen in benthic diatoms where it is suggested stochastic variation in OTU richness is a function of the decreased stability of smaller habitats \cite{bolgovics2016species}. \\

%Aquatic Microbial TARs
\noindent Aquatic habitats are some of the most studied in microbial biogeography due to the availability of insular water bodies and their range of spatial scales. An investigation into bacterial diversity in aquatic tree holes found a \textit{z} value comparable to macro-organisms (\textit{z} = 0.26) \cite{bell2005larger}. Antarctic cryoconite holes have also exhibited positive TARs on glaciers where biomass influx was limited, illustrating the significance of immigration rate \cite{darcy2018island}. Positive associations between habitat area and microbial OTU richness have also been reported for habitats as diverse as lakes \cite{battes2019species}, membrane bioreactors \cite{van2006bacterial} \cite{van2005island} and vertebrate bodies \cite{godon2016vertebrate}.\\

%Terrestrial Microbial TARs
\noindent Previous investigations into ectomycorrhizal fungi communities within 'tree island' root systems showed that total OTU richness increased significantly with size, although distance effects vary \cite{glassman2017theory} \cite{peay2007strong}. An investigation into bacterial and fungal diversity in a group of land-bridge islands showed OTU richness for both groups was positively correlated with area, but these same patterns were driven by different mechanisms \cite{li2020island}. The bacterial TAR was a produce by differences in habitat quality with island area, and the fungal TAR was driven by within--island dispersal limitation. Laboratory based experiments have supported the presence of soil microbial TARs as well as the influence of resource availability on OTU richness \cite{delgado2018experimentally}. Country and continent-scale patterns of pathogen diversity have also been shown to be a function of area and isolation \cite{jean2016equilibrium} \cite{cashdan2014biogeography}. In both terrestrial and aquatic systems microbial communities exhibit significant TARs. The varying mechanisms underlying these TARs warrant further investigation. \\

\noindent In this project I apply three modified versions of the model presented by Chisholm \textit{et al.,}: \\

\noindent {$\cdot$ \textbf{Classic Model}: Where per capita immigration rate is proportional to habitat area (e.g. in the case of aerial and directed dispersal species immigrating into a two-dimensional habitat)} \\
\noindent {$\cdot$ \textbf{Perimeter Model}: Where per capita immigration is proportional to habitat perimeter (e.g. in the case of water dispersed species immigrating into a two-dimensional habitat) } \\
\noindent {$\cdot$ \textbf{Depth Model}: Where per capita immigration rate is proportional to depth (e.g. in the case of species dispersing into a volume via its surface into a three-dimensional habitat) }\\

\noindent These models are applied to bacterial, archaeal and micro-eukaryote insular spatial data with the aim of testing whether there is a biphasic microbial TAR, as well as investigating the impact of habitat type and taxonomic group on critical area of transition between the deterministic and stochastic regimes. 
\chapter{Methods}
The model fitting process was validated by developing a simulation from which parameters of theta, migration rate and niches could be retrieved. A specifically designed laboratory experiment was commenced in January 2020 to test the model's theories in relation to microbial community dynamics. Due to the COV-19 pandemic the laboratory experiment was ended in March 2020. An alternative approach was devised, in which the model was fit to microbial species-area datasets compiled from the literature.

\section{Laboratory Experiment}

\subsection{Study area and sample collection}
\noindent

Soil was collected on site at Silwood Park, Berkshire, UK. Silwood Park comprises a variety of habitats including woodland, wetlands, heathland and formerly arable land \cite{CrawleyMichaelJ2005TfoB}. Soil types across the site consist of sandy or silty loam, with pH ranging from 4 to 6 \cite{LuckettKathryn2015Tbfr}. For this experiment soil was collected from a fallow field of acidic, sandy soil \cite{CrawleyMichaelJ2005TfoB}. Samples were acquired and sterilised during the first week of February. A total of 100 litres of soil was collected.

\subsection{Preparation of soil and mesocosms}
\noindent
The soil was homogenised by sieving and sterilising three times using an autoclave to destroy any present bacteria or fungi. We used five sizes of container: 0.2ml (PCR tube), 1.5ml (Eppendorf), 50ml (Fallon), 500ml and 5000ml. For each size of container 1, 2, 3, 4, 5 and maximum number of holes were made using a heated needle. Three replicates were made for each size of container and number of holes. The containers were then sterilised using the autoclave. The containers were filled with sterilised soil and sealed with an opaque lid. The containers were then buried on site, with sealed lids exposed and left to incubate for one month. 1.	If we use transparent covers for the tops of the tubes/containers this may create a gradient of light energy and moisture? Provide niches for heterotrophic and phototrophic microbes? Niche-based processes have been shown to be predominant in structuring observed aridity-related community patterns \cite{HuangMuke2019Iaas}. Abundance of predatory myxobacterial communities has also been correlated with temperature \cite{ZhouXiu‐Wen2014Mcia}.

\subsection{Soil properties measurements and climate data collection}
\noindent What is the pH of the soil where the samples were collected? pH of the soil where the samples were buried? pH has been show to affect microbial diversity \cite{GriffithsRobertI.2011Tbbo}.

\subsection{DNA extraction and sequence analysis}
\noindent After one month the containers were retrieved and DNA sampling was carried out the ZR Fungal/Bacterial DNA MiniPrep™ Isolation kit. Estimation of relative abundances of bacterial taxonomic groups was carried out using a previously defined PCR-based method \cite{FiererNoah2005Aosm}. Real-time PCR, or qPCR, allows for rapid quantitative assessment of soil microbial communities. I used 388 forward primer and 518 reverse primer as suggested for targeting all bacterial groups \cite{FiererNoah2005Aosm}. DNA samples from all 90 mesocosms were prepared for Illumina MiSeq 16S sequencing.

\section{Simulation}

\subsubsection{Overview}
This simulation is designed to mimic the process of island colonisation from a metacommunity. The colonisation process is constrained by migration rate and each island is characterised by number of niches and the size of each niche. The output of the simulation is the final community in each niche of each island, as well as a timeseries of species richness. 

\subsubsection{Metacommunity}
A metacommunity is generated at the beginning of the simulation using \textit{coalescence\_test} function, partially modified from a script provided by Dr James Rosindell. The function takes the input parameters: metacommunity size(\textit{J\_meta} = 50 000) and speciation rate (\textit{nu} = 0.001). Each run of the function produces 20 niche communities, each of size \textit{J\_meta}/20.  \bigskip

\noindent The function initialises a vector (\textit{lineages}) of length = \textit{J\_meta}/20 = \textit{niche\_size} with 1 as every value. An empty vector (\textit{abundances}) is initialised. The value of \textit{niche\_size} is given to \textit{N}. \textit{Theta} is calculated as \textit{nu*(niche\_size-1)/(1-nu)}. Then, while \textit{N $>$ 1}, a vector (\textit{linvect}) is created with values 1:length(\textit{niche\_size}). A random sample of \textit{linvect} is made (\textit{j}). A random decimal number is selected between 0 and 1 (\textit{randnum}). If \textit{rannum} is less than \textit{theta/(theta+N-1)}, then the value at \textit{lineages[j]} is appended to \textit{abundances}. Else, another random number (\textit{i}) is sampled from \textit{linvect}, excluding the last number selected. The values at \textit{lineages[i]} and \textit{lineages[j]} are summed and take the position of \textit{lineages[i]}. \textit{lineages[j]} is then removed from \textit{lineages}, so the vector is one value shorter. The value of \textit{N} is also decreased by 1. This repeats until \textit{N} = 1. The remaining value in the \textit{lineages} vector is added to \textit{abundances} and the function outputs a vector of simulated species abundances.\bigskip

\noindent At the end of the \textit{coalescence\_test} function, each of the 20 niches is assigned a letter type from A to T. A list of niche communities in the metacommunity is returned. It was chosen to generate no more than 20 niches within the metacommunity, because the simulation would not go beyond modelling 20 niches on an island. Any additional niches generated for the metacommunity would go unused. \bigskip

\noindent The \textit{coalescence\_test} function is incorporated into a second function (\textit{metacommunity}), that generates a vector of individuals from each niche abundance vector. For example, Niche A \textit{abundances}(5,4,2,2,1,1) would generate a community \textit{meta\$A}(1,1,1,1,1,2,2,2,2,3,3,4,4,5,6) where each unique number value represents a unique species.%\bigskip     

\subsubsection{Parameters}
The variable parameters of the simulation are: migration rate (range: 0.003-0.06), number of niches (range: 1-20), size of niches (range:1-20). Each unit of space was assumed to host one individual, therefore, number of niches x size of niches = area = size of island population. There are a total of 8000 condition combinations applied during the simulation (20 migration rates, 20 number of niches, 20 size of niches). 

\subsubsection{Simulation Logic}
At timestep \textit{i} an island is selected. The first niche on that island becomes the focal niche. An individual within that niche is chosen to die. With probability \textit{m} (migration rate), the dead individual is replaced with a randomly chosen propagule from the same niche type in the metacommunity (e.g. from metacommunity niche A to island niche A). With probability 1 - \textit{m}, the dead individual is replaced with a local propagule from the same niche. The simulation then moves to the next niche on that island. When all niches have been simulated for timestep \textit{i}, the simulation moves to the next island. When all islands have been simulated for timestep \textit{i}, the simulation moves to the next timestep \textit{i + 1}, and returns to the first island (Figure 1). The species richness for each niche is calculated, totaled across all niches for each island and stored at every 5000 timesteps.

\bigskip
 
 \begin{figure}[h!]
\centering
  \includegraphics[scale=0.4]{../../Other/neutral_flowchart2.png}
  \caption{Flowchart of simulation design}
  \label{fig:Flowchart}
\end{figure}


\subsubsection{High Performance Computing}
The metacommunity and simulation code are contained in \textit{ClusterSim.R}. \textit{ClusterSim.R} functions are sourced by \textit{ClusterCode.R} and given the input parameters: \textit{J\_meta} = 50000, \textit{nu} = 0.001, \textit{num\_m\_rates} = 20, \textit{max\_k\_num} = 20, \textit{max\_k\_size} = 20, \textit{wall\_time} = 1380, \textit{output\_file\_name} = output\_file\_name (where each simulation is given a unique file name "simulation\_timeseries\_\textit{i}". \textit{ClusterRun.sh} is used to run on the cluster, with a time limit of 24:00:00. \textit{wall\_time} is given as 1380 minutes (23 hours) within the function, to ensure all simulations are completed before the cluster run ends. 100 parallel simulation were run on the Imperial College London High Performance Computing service. This generated a total of 800000 islands, simulated for $>$ 30000 timesteps.   

\subsection{Data Preparation and Timeseries Plots}
The 100 simulation results were imported into \textit{DataPrep.R}. The data from each island was isolated and configured into a data frame with simulation number, migration rate, area, number of niches and number of species (\textit{SimModelFitData.csv}). A second data frame was generated for timeseries plotting, with simulation number, island number, migration rate, timestep and species richness timeseries for each island (\textit{SimTimeseriesPlotData.csv}).\bigskip

\noindent To ensure the simulation had run long enough for each island to reach dynamic equilibrium the species richness timeseries of simulations 25, 50 and 75 were plotted(\textit{TimeseriesPlot.R})(Figure 2).

\subsection{Analysis}\bigskip

\begin{verbatim}
chisholm_model <- function(area, theta, m0, K) {
  rho = 1
  K = K
  Js = area*rho
  J_stars = Js/K
  ms = m0/sqrt(area)
  gamma_stars = J_stars*ms/(1-ms)
  return(theta*(digamma(theta/K+gamma_stars*
  (digamma(gamma_stars+J_stars)-digamma(gamma_stars)))-digamma(theta/K)))
}
\end{verbatim}

\noindent An analysis script (\textit{Analysis.R}) imported the prepared data (\textit{SimModelFitData.csv}) for each island across all simulations (800000 islands). Model estimated species richnesses were generated by giving the island parameters (\textit{m0 = m*sqrt(area)}) and an estimated \textit{theta} (\textit{theta = 2*(niche\_size*K)*nu}, where \textit{niche\_size * K} = size of each niche in the metacommunity from which immigration events can occur times by the number of niches contributing to the island community i.e. number of niches on the island). The results of the simulation and those estimated by the \textit{chisholm\_function} (above) were bound together in a data frame. Mean species richness results for each combination of island area, migration rate and number of niches across all 100 simulations was calculated and stored.


\section{Data Collection and Analysis}

\subsection{Datasets}

\begin{table}[h!]
  \begin{center}
    \caption{Summary of datasets collected from the literature}
    \label{table1}
    \pgfplotstabletypeset[
      multicolumn names, % allows to have multicolumn names
      col sep=comma, % the seperator in our .csv file
      display columns/0/.style={
		column name=$Attribute$, % name of first column
		string type},  % use siunitx for formatting
      display columns/1/.style={
		column name=$Aqua$,
		column type={S},string type},
      display columns/2/.style={
		column name=$Terra$,
		column type={S},string type},
      display columns/3/.style={
		column name=$Total$,
		column type={S},string type},
      every head row/.style={
		before row={\toprule}, % have a rule at top
		after row={
			%\si{\ampere} & \si{\volt}\\ % the units seperated by &
			\midrule} % rule under units
			},
		every last row/.style={after row=\bottomrule}, % rule at bottom
    ]{summary_stats.csv} % filename/path to file
  \end{center}
\end{table}

In-lieu of being able to generate my own dataset, I compiled 36 datasets of microbial species/taxa-area relationships. A full description of each dataset and their authors can be found in the Supplementary Material section. These datasets included a range of taxonomic groups, including archaea, bacteria, fungi, algae, protazoa and pathogens. The datasets also included aquatic and terrestrial habitats, as well as in-situ and lab based investigations. For the majority of datasets, the number of observed species, strains, operational taxonomic units or phylogroups per "island" habitat was taken. Some studies only supplied diversity indexes (gamma diversity, phylogenetic diversity, faiths dp, simpsons index, chao 1, T-RFLP, viral prevalence). The estimated number of cells/individuals per unit area was also taken from each study, or estimated from the relevant literature if unavailable.     

\subsection{Model Fitting}

\begin{equation}
S=\theta\{\psi(\frac{\theta}{K}+\gamma(\psi(\gamma+J)-\psi(\gamma)))-\psi(\frac{\theta}{K})\}
\end{equation} 

For each dataset I found initial parameter estimates for \textit{m} and $\theta$. I then looped through all values of \textit{K} from 1 to the maximum number of taxa recorded on one "island" and fitted the Chisholm model (2.1) using NLLS fitting. From the fitting, the best-fit parameters for \textit{m} and $\theta$ were selected, along with their best fit \textit{K}. On "islands" where the estimated cell count per unit area was less than the value of \textit{K}, estimated species richness was constrained to equal number of individuals so as not to more species than individual microorganisms. %%%is this neccessary???

Values for rho were either taken from the original paper or from the literature.

\subsection{Critical Area}

\begin{equation}
ACrit=\frac{\theta(1 - m)(exp(K/\theta) - 1}{m\rho*log(1/m)}
\end{equation}

For each successful model fitting, the critical area of transition from a niche-structured regime to an extinction-colonisation equilibrium regime was estimated. The best-fit values of \textit{K}, \textit{m} and $\theta$ from the model fitting were given to the critical area equation (2.2). By finding the critical area of regime transition for the datasets, I was able to test the theory that the regime shift would occur at lower areas for "islands" that are less isolated. I then conducted a multiple regression analysis with critical area as dependent variable and habitat type (aquatic/terrestrial) and taxonomic group as categorical variables. 

\chapter{Results}

%Results. Describe your results in a logical order: this may not necessarily be the order in which you did the experiments. Briefly summarise the main results at the end of each main experiment or sequence of associated experiments. Do not duplicate results -- put a table or a graph but not both unless the two methods of presentation demonstrate different points of importance. You must refer appropriately to figures or tables in the text and remember to emphasise and perhaps quote significant results. In particular, think about:

%o What were the results of your hypothesis tests, in the order you describe them in the Methods?
{\texorpdfstring
%TTo verify the fitting procedure for each of the three models (Classic, Depth, Perimeter) I created three different model simulations. I ran the simulation data through three NLLS fitting scripts to retrieve the known parameters ($\theta$, as calculated from \textit{nu} and metacommunity size \textit{J*K}, \textit{m\textsubscript{0}} and \textit{K}). Here I have plotted the know parameters against those estimated by the model fitting procedure. Once the model fitting procedures are validated, I use them to fit the three models to my datasets, as well as calculating critical area (\textit{A\textsubscript{crit}}) between niche-structured regime and colonisation-extinction regime. Using critical area estimates I test the hypothesis that regime change will occur at lower areas for more motile species and less isolated habitats. 

\section\section{Simulation}

\begin{figure}[htbp]
\centering
\subfloat[The Classic Model]{\label{fig:a}\includegraphics[width=0.3\linewidth]{../Results/Simulation/NLLSPlotClassic5.pdf}}
\subfloat[The Depth Model]{\label{fig:b}\includegraphics[width=0.3\linewidth]{../Results/Simulation/NLLSPlotDepth6.pdf}}
\subfloat[The Perimeter Model]{\label{fig:c}\includegraphics[width=0.3\textwidth]{../Results/Simulation/NLLSPlotPeri8.pdf}}
\bigskip 


Simulation data point \tikz\draw[black,fill=black] (0,0) circle (.5ex); \;\;NLLS fit \textcolor{blue}{\rule{1.5cm}{1mm}} \textit{A\textsubscript{crit}}\;\; \textcolor{red}{\rule{0.1cm}{1mm}}\; \textcolor{red}{\rule{0.1cm}{1mm}}\; \textcolor{red}{\rule{0.1cm}{1mm}}\; \textcolor{red}{\rule{0.1cm}{1mm}}\\

\caption{NLLS fitting of simulation data. A) The Classic Model with true parameters $\theta$=13, \textit{m\textsubscript{0}}=0.05, \textit{K}=25 and estimated parameters $\theta$=13, \textit{m\textsubscript{0}}=0.05, \textit{K}=25. B) The Depth Model with true parameters $\theta$=18, \textit{m\textsubscript{0}}=0.06, \textit{K}=30 and estimated parameters $\theta$=19, \textit{m\textsubscript{0}}=0.05, \textit{K}=30. C) The Perimeters Model with true parameters $\theta$=32, \textit{m\textsubscript{0}}=0.7, \textit{K}=40 and estimated parameters $\theta$=35, \textit{m\textsubscript{0}}=0.6, \textit{K}=39.}
\label{fig:myfig}
\end{figure}

\noindent My results verified that the simulation and analytic formula are in agreement as expected. The three \textit{A\textsubscript{crit}} formulas for each of the three model variations give reasonable estimations.

\subsection{Classic Model}

\noindent The Classic Model fitted the simulated data well. Estimated parameters were slightly higher than the true parameters for $\theta$ and slightly lower for \textit{m\textsubscript{0}} and \textit{K}. There was a significant difference between the true and estimated parameters for $\theta$ (p=0.0009, 9 df) and \textit{m\textsubscript{0}} (p=0.0002, 9 df). There was no significant difference in parameters for \textit{K} (p=0.08, 9 df).  As \textit{K} parameters showed no significant difference, and there was only a small difference between \textit{m\textsubscript{0}} and $\theta$ parameters, the model fitting process is considered validated and fit for applying to empirical datasets.   \\

\begin{figure}[htbp]
\centering
\subfloat[$\theta$ values]{\label{fig:a}\includegraphics[width=0.3\linewidth]{AreaThetaResults.pdf}}
\subfloat[\textit{m\textsubscript{0}} values]{\label{fig:b}\includegraphics[width=0.3\linewidth]{Aream0Results.pdf}}
\subfloat[\textit{K} values]{\label{fig:c}\includegraphics[width=0.3\textwidth]{AreaKResults.pdf}}
\bigskip

Parameter differences data point \tikz\draw[black,fill=black] (0,0) circle (.5ex); \;\; 1:1 line \textcolor{black}{\rule{1.5cm}{1mm}}\\

\caption{The true parameters values of $\theta$ (A), \textit{m\textsubscript{0}} (B) and \textit{K} (C) were simulated and the results fitted using the Classic Model analytical NLLS fitting procedure to get the parameters back. True and estimated values for the three fitted parameters are plotted above. Fittings of the Classic Model to simulated data returned mean R\textsuperscript{2} = 0.99, adjusted R\textsuperscript{2} = 0.99.}
\label{fig:myfig}
\end{figure}

\begin{table}[h!]
  \begin{center}
    \caption{Comparison between true and estimated mean parameters across 200 Classic Model simulations clustered into 10 groups where parameter values ($\theta$, \textit{m\textsubscript{0}}, \textit{K}) were the same for each simulation group with varying areas.}
    \label{table5}
    \pgfplotstabletypeset[
      multicolumn names, % allows to have multicolumn names
      col sep=comma, % the seperator in our .csv file
      display columns/0/.style={
		column name=$Parameter$, % name of first column
		string type},  % use siunitx for formatting
      display columns/1/.style={
		column name=$True$,
		column type={S},string type},
      display columns/2/.style={
		column name=$Estimated$,
		column type={S},string type},
      display columns/3/.style={
		column name=$Difference$,
		column type={S},string type},
           every head row/.style={
		before row={\toprule}, % have a rule at top
		after row={
			%\si{\ampere} & \si{\volt}\\ % the units seperated by &
			\midrule} % rule under units
			},
		every last row/.style={after row=\bottomrule}, % rule at bottom
    ]{ClassicParam.csv} % filename/path to file
  \end{center}
\end{table}



\subsection{Depth Model}

\noindent The Depth Model fitted the simulated data well. Estimated values of $\theta$ were higher, \textit{K} and \textit{m\textsubscript{0}} values were lower than the true parameters (Table 3.2). There was a significant difference between the true and estimated values for $\theta$ (p=0.0005, 9 df) and \textit{m\textsubscript{0}} (p=0.005, 9 df), but there was no significant difference for \textit{K} (p=0.168, 9 df). As \textit{K} showed no significant difference, and the difference in estimated $\theta$ and \textit{m\textsubscript{0}} values were low, the model fitting process is considered validated and fit for applying to empirical datasets.  

\begin{figure}[htbp]
\centering
\subfloat[$\theta$ values]{\label{fig:a}\includegraphics[width=0.3\linewidth]{DepthThetaResults.pdf}}
\subfloat[\textit{m\textsubscript{0}} values]{\label{fig:b}\includegraphics[width=0.3\linewidth]{Depthm0Results.pdf}}
\subfloat[\textit{K} values]{\label{fig:c}\includegraphics[width=0.3\textwidth]{DepthKResults.pdf}}
\bigskip

Parameter differences data point \tikz\draw[black,fill=black] (0,0) circle (.5ex); \;\; 1:1 line \textcolor{black}{\rule{1.5cm}{1mm}}\\

\caption{The true parameters values of $\theta$ (A), \textit{m\textsubscript{0}} (B) and \textit{K} (C) were simulated and the results fitted using the Depth Model analytical NLLS fitting procedure to get the parameters back. True and estimated values for the three fitted parameters are plotted above. Fittings of the Depth Model to simulated data returned mean R\textsuperscript{2} = 0.99, adjusted R\textsuperscript{2} = 0.99.}
\label{fig:myfig}
\end{figure}

\begin{table}[h!]
  \begin{center}
    \caption{Comparison between true and estimated mean parameters across 200 Depth Model simulations clustered into 10 groups where parameter values ($\theta$, \textit{m\textsubscript{0}}, \textit{K}) were the same for each simulation group with varying areas.}
    \label{table5}
    \pgfplotstabletypeset[
      multicolumn names, % allows to have multicolumn names
      col sep=comma, % the seperator in our .csv file
      display columns/0/.style={
		column name=$Parameter$, % name of first column
		string type},  % use siunitx for formatting
      display columns/1/.style={
		column name=$True$,
		column type={S},string type},
      display columns/2/.style={
		column name=$Estimated$,
		column type={S},string type},
      display columns/3/.style={
		column name=$Difference$,
		column type={S},string type},
           every head row/.style={
		before row={\toprule}, % have a rule at top
		after row={
			%\si{\ampere} & \si{\volt}\\ % the units seperated by &
			\midrule} % rule under units
			},
		every last row/.style={after row=\bottomrule}, % rule at bottom
    ]{DepthParam.csv} % filename/path to file
  \end{center}
\end{table} 


\subsection{Perimeter Model}


\begin{figure}[htbp]
\centering
\subfloat[$\theta$ values]{\label{fig:a}\includegraphics[width=0.3\linewidth]{PeriThetaResults.pdf}}
\subfloat[\textit{m\textsubscript{0}} values]{\label{fig:b}\includegraphics[width=0.3\linewidth]{Perim0Results.pdf}}
\subfloat[\textit{K} values]{\label{fig:c}\includegraphics[width=0.3\textwidth]{PeriKResults.pdf}}
\bigskip

Parameter differences data point \tikz\draw[black,fill=black] (0,0) circle (.5ex); \;\; 1:1 line \textcolor{black}{\rule{1.5cm}{1mm}}\\

\caption{The true parameters values of $\theta$ (A), \textit{m\textsubscript{0}} (B) and \textit{K} (C) were simulated and the results fitted using the Perimeter Model analytical NLLS fitting procedure to get the parameters back. True and estimated values for the three fitted parameters are plotted above. Fittings of the Perimeter Model to simulated data returned mean R\textsuperscript{2} = 0.99, adjusted R\textsuperscript{2} = 0.99.}
\label{fig:myfig}
\end{figure}

\begin{table}[h!]
  \begin{center}
    \caption{Comparison between true and estimated mean parameters across 200 Perimeter Model simulations clustered into 10 groups where parameter values ($\theta$, \textit{m\textsubscript{0}}, \textit{K}) were the same for each simulation group with varying areas.}
    \label{table5}
    \pgfplotstabletypeset[
      multicolumn names, % allows to have multicolumn names
      col sep=comma, % the seperator in our .csv file
      display columns/0/.style={
		column name=$Parameter$, % name of first column
		string type},  % use siunitx for formatting
      display columns/1/.style={
		column name=$True$,
		column type={S},string type},
      display columns/2/.style={
		column name=$Estimated$,
		column type={S},string type},
      display columns/3/.style={
		column name=$Difference$,
		column type={S},string type},
           every head row/.style={
		before row={\toprule}, % have a rule at top
		after row={
			%\si{\ampere} & \si{\volt}\\ % the units seperated by &
			\midrule} % rule under units
			},
		every last row/.style={after row=\bottomrule}, % rule at bottom
    ]{PeriParam.csv} % filename/path to file
  \end{center}
\end{table}

\noindent The Perimeter Model fitted the simulated data well. Estimated parameters for $\theta$ were slightly higher than true parameters, whilst \textit{m\textsubscript{0}} and \textit{K} were slightly lower than the true parameters (Table 3.3). There was significant difference between the true and estimated values for $\theta$ (p=0.0002, 9 df), \textit{m\textsubscript{0}} (p=5x10\textsuperscript{5}, 9 df) and \textit{K} (p=0.04, 9df). Despite the significant difference between the estimated and true values, the differences are small and the model fitting process is considered validated and fit for applying to empirical datasets.   

\section{Model Fitting}

\subsection{Non-Linear Least Squares Fitting}


\noindent 50 of the 57 datasets exhibited a positive TAR and were used for the NLLS fitting. Of the 50 datasets 26 failed to achieve adjusted R\textsuperscript{2} scores of between 0 and 1. These datasets were excluded from further analysis (see Supplementary Materials, Figure 7.2). 

\begin{table}[h!]
  \begin{center}
    \caption{The mean R\textsuperscript{2} and adjusted R\textsuperscript{2} results for each model (Classic, Depth, Perimeter) after being successfully fitted to 24 empirical datasets.}
    \label{table5}
    \pgfplotstabletypeset[
      multicolumn names, % allows to have multicolumn names
      col sep=comma, % the seperator in our .csv file
      display columns/0/.style={
		column name=$Model$, % name of first column
		string type},  % use siunitx for formatting
      display columns/1/.style={
		column name=$R\textsuperscript{2}$,
		column type={S},string type},
      display columns/2/.style={
		column name=$Adj R\textsuperscript{2}$,
		column type={S},string type},
           every head row/.style={
		before row={\toprule}, % have a rule at top
		after row={
			%\si{\ampere} & \si{\volt}\\ % the units seperated by &
			\midrule} % rule under units
			},
		every last row/.style={after row=\bottomrule}, % rule at bottom
    ]{RResults.csv} % filename/path to file
  \end{center}
\end{table}

\noindent All three models had similar mean R\textsuperscript{2} and adjusted R\textsuperscript{2} scores and fit the data moderately well (Table 3.4). The Classic Model was best-fit for 1 dataset, Depth and Perimeter were best for 2 each and the rest of the datasets were either best described by both Classic and Depth or all of the models (Table 3.5).  \\

\begin{table}[h!]
  \begin{center}
    \caption{The best-fit models (Classic, Depth, Perimeter) by highest adjusted R\textsuperscript{2} value for each empirical dataset (note some datasets had equal adjusted R\textsuperscript{2} values for two or more models). }
    \label{table5}
    \pgfplotstabletypeset[
      multicolumn names, % allows to have multicolumn names
      col sep=comma, % the seperator in our .csv file
      display columns/0/.style={
		column name=$Models$, % name of first column
		string type},  % use siunitx for formatting
      display columns/1/.style={
		column name=$Best Fit$,
		column type={S},string type},
           every head row/.style={
		before row={\toprule}, % have a rule at top
		after row={
			%\si{\ampere} & \si{\volt}\\ % the units seperated by &
			\midrule} % rule under units
			},
		every last row/.style={after row=\bottomrule}, % rule at bottom
    ]{BestModel.csv} % filename/path to file
  \end{center}
\end{table}

\noindent The best model fits had mean R\textsuperscript{2} = 0.49 and mean adjusted R\textsuperscript{2} = 0.41 with standard deviation 0.28 and range 0.01 - 0.96. The median value of $\theta$ was 8, with a range of 0.28 -- 159709. The median value of \textit{m\textsubscript{0}} was 2.17 x10\textsuperscript{-9} with a range of 4.97 x 10\textsuperscript{-16} -- 0.56. The median value for \textit{K} was 7, with range 1 -- 424. There was no correlation between the best fitted-values of the four parameters (Table 3.6). \\

\begin{table}[h!]
  \begin{center}
    \caption{p-values of correlations between the four model parameters ($\theta$, \textit{m\textsubscript{0}}, \textit{K}, $\rho$) that show no correlation}
    \label{table5}
    \pgfplotstabletypeset[
      multicolumn names, % allows to have multicolumn names
      col sep=comma, % the seperator in our .csv file
      display columns/0/.style={
		column name=$Parameter$, % name of first column
		string type},  % use siunitx for formatting
      display columns/1/.style={
		column name=$K$,
		column type={S},string type},
      display columns/2/.style={
		column name=$Theta$,
		column type={S},string type},
        display columns/3/.style={
		column name=$m\textsubscript{0}$,
		column type={S},string type},
	 display columns/4/.style={
		column name=$rho$,
		column type={S},string type},
           every head row/.style={
		before row={\toprule}, % have a rule at top
		after row={
			%\si{\ampere} & \si{\volt}\\ % the units seperated by &
			\midrule} % rule under units
			},
		every last row/.style={after row=\bottomrule}, % rule at bottom
    ]{Spearmansrank_p_corr.csv} % filename/path to file
  \end{center}
\end{table}

%How often was the power-law model a better fit to the data than the three models?
\noindent The power-law model had the same number of successful fittings as the Classic, Depth and Perimeter models. After removing failed fits the mean \textit{z} value was 0.16. The power-law model performed more poorly than the other three models (R\textsuperscript{2}=0.47, adjusted R\textsuperscript{2}=0.38) (see Supplementary Materials, Table 7.3). AIC scores indicated that the power-law model was not a more parsimonious model than the Classic, Depth or Perimeter models relative to model fit for any of the datasets (see Supplementary Materials, Table 7.4). The Classic and Depth models were significantly better than the power-law model for 9 datasets each. The Perimeter model was a better fit than the power-law model for 5 datasets.   
  
\section{Critical Area}
{\texorpdfstring
Off the five habitat types (terrestrial, riverine, lacustrine, plant and machine) and six taxonomic groups (algae, archaea, bacteria, fungi, pathogens and protozoa), the riverine habitat and archaea group did not have any successful fittings and are excluded from the following analysis.\\


\begin{figure}[htbp]
\centering
\subfloat[Dataset 45, bacteria in biomembrane reactors]{\label{fig:a}\includegraphics[width=0.45\linewidth]{../Results/ClassicNLLSPlot39.pdf}}\qquad
\subfloat[Dataset 44, fungi in plant root soil]{\label{fig:b}\includegraphics[width=0.45\linewidth]{../Results/PeriNLLSPlot38.pdf}}\\
\subfloat[Dataset 46, bacteria in tree holes (log area plotted with depth)]{\label{fig:c}\includegraphics[width=0.45\textwidth]{../Results/DepthNLLSPlot40.pdf}}\qquad%
\subfloat[Dataset 46, bacteria in tree holes (log volume plotted with OTU richness)]{\label{fig:d}\includegraphics[width=0.45\textwidth]{../Results/DepthNLLSPlot2_40.pdf}}%
\bigskip

Simulation data point \tikz\draw[black,fill=black] (0,0) circle (.5ex); \;\;NLLS fit \textcolor{blue}{\rule{1.5cm}{1mm}} \;\;Power-Law fit \textcolor{green}{\rule{1.5cm}{1mm}}\;\; \textit{A\textsubscript{crit}/A\textsubscript{vol}}\;\; \textcolor{red}{\rule{0.1cm}{1mm}}\; \textcolor{red}{\rule{0.1cm}{1mm}}\; \textcolor{red}{\rule{0.1cm}{1mm}}\; \textcolor{red}{\rule{0.1cm}{1mm}}\\

\caption{A) Best-fit Classic Model for dataset 45, bacteria in biomembrane reactors. Red line indicates \textit{A\textsubscript{crit}}, blue line indicates NLLS fit and green line indicates power-law fit (NLLS fit: R\textsuperscript{2}=0.96, adjusted R\textsuperscript{2}=0.88, $\theta$=9, \textit{m\textsubscript{0}}=4.97 x 10\textsuperscript{-16}, \textit{K}=7, power-law fit: R\textsuperscript{2}=0.94, adjusted R\textsuperscript{2}=0.82, \textit{z}=0.27, \textit{c}=0.88). B) Best-fit Perimeter Model for dataset 44, fungi in plant soil (NLLS fit: R\textsuperscript{2}=0.85, adjusted R\textsuperscript{2}=0.77, $\theta$=5, \textit{m\textsubscript{0}}=6.15 x 10\textsuperscript{-11}, \textit{K}=2, power-law fit: R\textsuperscript{2}=0.87, adjusted R\textsuperscript{2}=0.78, \textit{z}=0.18, \textit{c}=0.64). C) Best-fit Depth Model for dataset 46, bacteria in freshwater treeholes. The size of the black circles represents increasing OTU richness at that corresponding depth (x-axis) and log area (y axis) (R\textsuperscript{2}=49, adjusted R\textsuperscript{2}=0.40, $\theta$=8, \textit{m\textsubscript{0}}=3.75 x 10\textsuperscript{-9}, \textit{K}=6, power-law fit: R\textsuperscript{2}=0.46, adjusted R\textsuperscript{2}=0.38, \textit{z}=0.33, \textit{c}=1.74). Where the red line passes through depth and area space is where \textit{A\textsubscript{crit}} occurs. D) Dataset 46 plotted as log Volume by OTU richness to illustrate the model fit and log critical volume (A\textsubscript{vol})}
\label{fig:myfig}
\end{figure}

\noindent The log \textit{A\textsubscript{crit}} data were not normally distributed with non-homogenous variances. Despite the violation of normality I have proceeded with the multiple regression analysis, although interpretation of results will take this into consideration. \\

\noindent Initial multiple regression revealed that the model was a poor fit to the data (R\textsuperscript{2}=0.39, adjusted R\textsuperscript{2}=0.05, p=0.374) and neither categorical variable was significant in predicting log \textit{A\textsubscript{crit}} (habitat type p=0.1759, taxonomic group p=0.6402). A plot of the model indicated that there was an outlying data point. After removing the outlying data point the model was significant in describing the data (R\textsuperscript{2}=0.62, adjusted R\textsuperscript{2}=0.44, p=0.02). Taxonomic group became weakly significant in predicting log \textit{A\textsubscript{crit}} (p=0.0187), habitat type did not (p=0.097). After removing habitat type as a variable the model was a similar fit to the data but more significant (R\textsuperscript{2}=0.55, adjusted R\textsuperscript{2}=0.45, p=0.004).\\

\noindent Multiple regression including taxonomic group upheld the prediction that \textit{A\textsubscript{crit}} would occur at lower areas for more motile OTUs as bacteria show the lowest log\textit{A\textsubscript{crit}} estimate and host-dependent pathogens show the highest (Table 3.7). \\

\noindent There was a large variation in mean log \textit{A\textsubscript{crit}} between habitats and taxonomic groups. Terrestrial habitats showed the highest mean log \textit{A\textsubscript{crit}} (27.33), whilst machine habitats showed the lowest (4.66) (Figure 3.6). Pathogens exhibited the largest mean log \textit{A\textsubscript{crit}} for taxonomic groups (55.15), with bacteria having the lowest (4.06) (Figure 3.6). \\

\begin{figure}[htp]

\centering
\includegraphics[width=.5\textwidth]{BoxplotTotalACritArch.pdf}\hfill
\includegraphics[width=.5\textwidth]{BoxplotTotalACritTaxa.pdf}\hfill

\hspace{10pt}\textbf{Habitat Type} \hspace{110pt} \textbf{Taxonomic Group}

\caption{log \textit{A\textsubscript{crit}} by habitat type and taxonomic group after removing anomalous result}
\label{fig:figure8}

\end{figure}


}

\begin{table}[h]
\begin{center}
    \caption{Table showing the results of multiple regression analysis of estimated effect of taxonomic group only on log \textit{A\textsubscript{crit}}}
    \label{crouch}
    \begin{tabular}{  l  p{1.5cm} p{3cm}  p{1.5cm}}
        \toprule
\textbf{Variable} 
&\textbf{Estimate}      
& \textbf{95\% CI}
& \textbf{p-value}   \\\midrule
intercept (algae)
&20.56
& [5.07, 36.07]
& 0.0121 \\\hline
taxonomic group
&
& 
&0.004 \\\hline
bacteria
&-16.500
& [-35.12, 2.12]
&0.0791 \\\hline
fungi
&-6.222 
& [-27.01, 14.56]
& 0.5373 \\\hline
pathogens
&34.587
& [7.75, 61.42]
& 0.0144 \\\hline
protozoa
&6.879 
& [-16.79, 30.55]
& 0.5491  \\
        \bottomrule
    \end{tabular}
    \end{center}
\end{table}









\chapter{Discussion}

%DONE
I have presented three variations of the Chisholm model \cite{chisholm2016maintenance} that take into account varying habitats and immigration routes and have successfully fit all three to microbial TAR data. The relatively equal success of the three model variations (Classic, Depth, Perimeter) suggests that immigration route is not a significant factor in defining microbial TARs (see Supplementary Materials, Table 7.5).  Microorgansims can cross oceanic, glacial and global barriers through a variety of dispersal mechanisms \cite{rosselli2015microbial} \cite{darcy2018island} \cite{kleinteich2017pole}. Microbial OTUs utilise a variety of immigration routes when entering a new environment. No significant accuracy was lost in assessing three-dimensional habitats using the two-dimensional models (Classic, Perimeter), suggesting habitat depth did not affect OTU richness as strongly as area, where area acts as immigration portal into the three-dimensional habitat. Algae and bacteria have shown negative correlations between OTU richness and depth \cite{battes2019species} \cite{turner2017microbial}. Nutrient-poor, low-energy environments may accompany increasing habitat depth. The immigration portal may be characterised by a nutrient-rich, high-energy stratification that is a more potent predictor of OTU richness than depth. \\

%%%%talk about the power law model
%DONE
\noindent The phenomenological power-law model did not perform significantly better than the mechanistic Classic, Depth and Perimeter models, according to the AIC measure of parsimony, relative to model fit (see Supplementary Materials, Table 7.4). This indicates that the model parameters ($\theta$, \textit{m\textsubscript{0}}, \textit{K}, $\rho$) were useful in describing TARs, supporting the hypothesis that OTU richness is influenced by the parameters, rather than being a constant power of area. The mean slope of positive TARs across these datasets was comparable to macroorganisms (\textit{z}=0.16) and higher than those previously reported for microbial taxa \cite{rosenzweig1995species} \cite{green2004spatial} (see Supplementary Materials, Table 7.3). These observations show habitat area has a relatively strong influence on OTU richness.  \\

%DONE
\noindent Successfully fitted datasets exhibit both the classic MacArthur and Wilson \cite{MacArthurRobertH1967Ttoi} biogeographic pattern and the SIE. The results demonstrate that some microbial communities are constrained by niche-structured regimes at smaller areas where immigration is low, before transitioning to colonisation-extinction balance regimes at larger areas where immigration is high. This lends support to the theory that microbial species are not ubiquitous and unlimited in dispersal, that they can be limited by habitat heterogeneity, resource availability and dispersal barriers, but this is not a ubiquitous pattern with over 50\% of the datasets failing to be fit by the model. \\

%DONE
\noindent Many datasets with positive TARs could not be successfully fit with the power-law, Classic, Depth or Perimeter models (see Supplementary Materials, Figure 7.2). Despite positive \textit{z} values, confidence intervals included zero and were not statistically significant. Stochastic variation between data points inhibited identification of significant TARs (see Supplementary Materials, Figure 7.2 a \& b). The majority of failed fits were aquatic habitats and may be due to uncertainty in the spatial sample regime of a heterogenous habitat. In order to elucidate these patterns it might be useful to take a stratified approach. Some failed fits had too few data points in comparison to the number of model parameters, producing low adjusted R\textsuperscript{2} values (see Supplementary Materials, Figure 7.2 b). Microbial TARs may also be undetected due to the disparity between sample and community sizes meaning rare taxa are missed \cite{woodcock2006taxa}.\\ 

%DONE
\noindent This project is, to the best of my knowledge, the first attempt to apply a biphasic mechanistic TAR model to microbial data. The model demonstrates that when niche diversity increases slowly or remains constant and immigration increases quickly with area, a biphasic TAR is produced. At an \textit{A\textsubscript{crit}} specific to that habitat and taxonomic group, the TAR will transition from deterministic to a stochastic mechanisms. I hypothesised that \textit{A\textsubscript{crit}} would be lower where immigration is higher (i.e. for more motile OTUs and less isolated habitats). My analysis indicated that taxonomic group was significant in predicting \textit{A\textsubscript{crit}}, while habitat type was not. Taxonomic group is significant in predicting \textit{A\textsubscript{crit}} as taxa are constrained (or liberated) by their life stages (activity and dormancy) and dispersal methods. According to this data isolated habitats present no significant dispersal barriers to microorganisms as a whole, although their relative accessibility varies between taxonomic groups. Overall, after removing the outlying datapoint and removing habitat type as an explanatory variable, the model accounts for nearly half of the variation in log \textit{A\textsubscript{crit}} using broad taxonomic groups.\\

%DONE
\noindent My analysis indicated that pathogenic OTUs had overwhelmingly higher \textit{A\textsubscript{crit}} values (mean 9.43 x 10\textsuperscript{30} cm\textsuperscript{2}), suggesting they are more constrained by resource availability and dispersal barriers. This may be due to their dependence on host species, although this will be directly related to the motility and sociability of their hosts. The two datasets used in this study quantify human pathogen richness on `true' islands \cite{jean2016equilibrium}. Human pathogen OTU richness is negatively correlated with disease control efforts \cite{dunn2010global}. I suggest that global mitigation strategies such as behavioural change, medicine and vaccination mean pathogens face considerable dispersal barriers that limit immigration and constrain them to niche-structured spatial regimes over larger areas  \cite{nicolaides2020hand} . \\

%DONE
\noindent Bacterial OTUs exhibited the lowest mean \textit{A\textsubscript{crit}} value (3.02 x 10\textsuperscript{3} cm\textsuperscript{2}). The small size of bacteria allows them to dispersal more freely than size-limited macroorganisms \cite{martiny2006microbial}. They may also overcome dispersal limitation through dormancy and enormous population sizes \cite{LoceyKennethJ2010Stbw} \cite{fenchel2004ubiquity}. Bacteria have a variety of ecological traits that allow them to move freely and access isolated habitats, thus they transition to stochastic TAR mechanisms at lower areas. \\
 
 %DONE
\noindent Fungi also showed low \textit{A\textsubscript{crit}} values (mean 1.72 x 10\textsuperscript{16} cm\textsuperscript{2}). Mycorrhizal fungi that share beneficial associations with plant roots, have large spores that immigration slowly through soil \cite{bueno2019arbuscular}, however, the close proximity of potential host plants might mitigate low fungal motility. For other fungal groups, long distance spore dispersal is facilitated by meteorological, biotic and anthropogenic vectors \cite{golan2017long}. Fungal sporulation allows taxa to overcome local and regional barriers, thus contributing to the low \textit{A\textsubscript{crit}} values seen in these datasets. \\

%DONE
\noindent Algae (mean 2.56 x 10\textsuperscript{19} cm\textsuperscript{2}) and protozoa (mean 7.43 x 10\textsuperscript{23} cm\textsuperscript{2}) exhibited similar broad, midrange \textit{A\textsubscript{crit}} values which may be due issues of sampling in spatially heterogenous aquatic environments. Issues of taxonomic classification, particularly for protists, may contribute to varying estimations of diversity \cite{foissner2006biogeography}. Whilst it seems algae and protists transition from deterministic to stochastic mechanisms in the midrange of areas, further investigation is needed to discern a true pattern within the wide range of \textit{A\textsubscript{crit}} values estimated. The multiple regression model coefficients (Table 3.7) broadly confirm the overall taxonomic results. \\

%DONE
\noindent As the multiple regression analysis showed that habitat type was non-significant in predicting \textit{A\textsubscript{crit}} I cannot assess the relative isolation of habitats or how they may affect \textit{A\textsubscript{crit}}. The non-significance of habitat type, despite marked differences in the mean \textit{A\textsubscript{crit}} is due to the large, overlapping estimate ranges. It is interesting however to look at mean \textit{A\textsubscript{crit}} values for each habitat, as a sign post towards what may be found with a more comprehensive dataset. Terrestrial habitats show the highest mean \textit{A\textsubscript{crit}} values (2.36 x 10\textsuperscript{30} cm\textsuperscript{2}). This may be due to immigration via an accidental vector being limited to aerial species that can reach land islands. Passive immigration to land islands relies on stochastic success which may limit dispersal, although fungal and bacteria OTU richness on land islands has been shown be unaffected by isolation \cite{li2020island}. \\

%DONE
\noindent Lacustrine habitats exhibited low mean \textit{A\textsubscript{crit}} values (1.28 x 10\textsuperscript{19} cm\textsuperscript{2}) suggesting immigration to these habitats is high. Aquatic taxa utilise a variety of dispersal mechanisms between habitats, including dispersal via insects and waterfowl \cite{stewart1966dispersal}. It may be easier for microbial OTUs to colonise inland lacustrine environments where animal activity increases the probability of vector transport. There may be higher rates of passive transport to lacustrine environments due to the interconnectivity of rivers and streams that empty into watershed areas, filling lakes and ponds.  

%DONE
\noindent Plant habitats also have low mean \textit{A\textsubscript{crit}} values (2.01 x 10\textsuperscript{4} cm\textsuperscript{2}). For many symbiotic plant-microbe species relationships, plant seeds are already inoculated with associated microbial taxa on dispersal \cite{ho2017plant}. Thus dispersal barriers between plant and microbes are removed, contributing to low \textit{A\textsubscript{crit}} values. Many plant communities are comprised of the same species in close proximity, providing ready access to source populations and increasing immigration. \\

%DONE
\noindent Four of the six best-fit datasets were for bacteria in machine habitats \cite{van2006bacterial} \cite{van2005island}. It may be the strong TAR is a function of isolation, relative to natural habitats. When constrained by immigration more prominent and easily quantifiable TAR patterns arise. The model fitting process supports this by estimating extremely low immigration rates for machine habitats. Despite this, machine habitats had the lowest mean \textit{A\textsubscript{crit}} (6.56 x 10\textsuperscript{3} cm\textsuperscript{2}). \textit{A\textsubscript{crit}} is not only affected by immigration as in my primary hypothesis, but can also be affected \textit{K}, $\rho$ and $\theta$. In the fitted model the low \textit{A\textsubscript{crit}} for machine habitats in spite of their isolation is caused by the low \textit{K} values of homogenous, man-made environments, a characteristic of these unusual habitats that warrants further investigation. \\

%DONE
\noindent The reason for the lack of successful fittings for riverine habitats is due to the low number of data points that forced low adjusted R\textsuperscript{2} scores, despite initial good fits (see Table 7.1, Supplementary Material). Only one dataset included archaeal TARs and no significant relationship between area and OTU richness was found. This is likely due to the importance of environmental filtering for extremophile OTUs in soda lakes \cite{LanzenAnders2013SPaE}. It is interesting to note other datasets removed from the fitting process due to a lack of positive TARs. These included, fungi in the Antarctic cryoconite holes of two glaciers where extreme biomass influx negated observable TARs \cite{darcy2018island}. Inappropriate diversity metrics and spatial scaling may have led to undetectable root-symbiotic fungi TARs \cite{davison2018microbial}. Fungi OTU richness did not increase with area on submerged leaves due to a lack of energy increase with corresponding area as expressed by the species-energy theory \cite{FeinsteinLarryM2012Tran}  \cite{wright1983species}. TARs may not have arisen in protozoan communities due to a failure to reach equilibrium (dataset 55) \cite{henebry1980effect}. It is clear that the factors affecting microbial TARs are diverse and each habitat/taxa pairing may require unique assessment. \\ 

%DONE
%What was the anomalous result? Why was it anomalous?
\noindent The anomalous result removed from analysis concerned pathogenic bacterial OTU richness on `true' geographic islands \cite{jean2016equilibrium}. The model was a poor fit to the data (R\textsuperscript{2}=0.23, adjusted R\textsuperscript{2}=0.18) and it's likely the error associated with estimating density for pathogenic bacteria over such large geographic scales lead to poor estimations of the remaining parameters and excessive \textit{A\textsubscript{crit}}.   \\

%DONE
%What are the caveats that apply to this study? (Leave out caveats that apply to all studies.)
\noindent Whilst the data indicate that habitat type is non-significant in predicting \textit{A\textsubscript{crit}}, the large range in \textit{A\textsubscript{crit}} values suggests there many be too few data points to discern a significant pattern. I also encountered challenges when trying to compare studies that used a variety of quantification techniques. Microbial OTUs inhabit three-dimensional habitats and whilst steps have been taken to account for this there is more work to be done to incorporating this fully. In an extension of this model I would consider each stratification of a habitat separately, to account for spatial heterogeneity. Volume has been shown to be more accurate in quantifying microbial TARs \cite{van2006bacterial}. It would be useful to further modify the model to explicitly incorporate volume and \textit{V\textsubscript{crit}} across datasets, as nearly all of them concern habitats within a volume even though often only area data is available. Here I have used area with a depth metric (Depth Model) which suggests the habitat maintains the same area for the full depth, whereas natural habitats rarely take this shape and this reduces the accuracy of my results. \\

%DONE
\noindent Another issue I encountered was $\rho$ estimations as direct counts are rarely given for microbial OTUs. Estimations were made in various ways, using gene sequence numbers or proxy papers. It would be beneficial to develop more robust methods for estimating $\rho$ as data from proxy papers introduces error. A broad scale experiment to quantify microbial TARs in a laboratory, where data specific to the needs of these models could be collected would provide a more vigorous assessment of the models applicability to microbial TARs. \\

%DONE
\noindent When validating my fitting procedures there was small but significant error between true and estimated parameters. Parameter ranges of speciation rate, \textit{m\textsubscript{0}} and \textit{K} were not inferred from microbial ecological theory, but were selected for ease of computation. A more thorough exploration of the parameter space, with ecologically relevant parameter ranges, to further validate the fitting procedure and the areas of parameter space where there may be errors in fitting would be desirable in further work. \\

%DONE
\noindent There remains to be a thorough synthesis between biogeography and microbial ecology. Here I have gone some way to evaluate the influence of immigration on microbial TARs, however more work is needed to examine dispersal barriers. Dormancy is a widespread microbial response that may allow OTUs to overcome dispersal barriers and increase immigration to new habitats. However, it is a slow, passive process that will not necessarily lead the individual to a viable habitat \cite{LoceyKennethJ2010Stbw}. To fully elucidate the interplay of microbial ecology and biogeographic patterns, work is needed to incorporate dormancy as a biogeographic response.  \\         

%What are the implications of this work?
%DONE
\noindent An implication of this work is that if we can identify the niches within a habitat and the taxonomic groups that tend to occupy those niches, we may be able to better predict OTU richness at a range of spatial scales. This presents a complex challenge that requires the integration of environmental variables and habitat stratification. If these challenges could be overcome, it would be a useful tool in predicting colonisation of new habitats such as soil exposed by glacier retreat, helping us model the biogeochemical processes of colonisation. \\      

%A nice wrap-up, emphasising how this study in this system is of interest to people who work on other things, or other systems.
\noindent This study has demonstrated that microbial communities in isolated `island' habitats can be subject to niche-structured regimes, before a critical area of transition to colonisation--extinction regimes. I have shown that taxonomic group is significant in predicting critical area, but habitat type is not. The overwhelming number, complexity and importance of microbial life illustrates the need to understand their biogeographic patterns. I hope that my study will lead to further research into deterministic and stochastic mechanisms in microbial biogeography, as well as the importance of taxonomic group on the relative influence of these processes. The synthesis of microbial ecology and biogeography will be of increasing interest as climate change alters habitats, extending microbial ranges and leading to climate feedback loops. Microbial biogeography is an essential area of study in our global challenge to predict and mitigate the impacts of climate change. Everything is \textit{not} everywhere, and everything is changing.             
\chapter{Data and Code Availability}

Name a data (e.g., Dropbox, FigShare, Zenodo, etc) and a code (e.g., Dropbox, GitHub, etc.) archive from where the data and code can be obtained that will allow replication of your results. The code may be in the form of a single script file. You will be taught the principles of reproducible analyses in the R week of your coursework. If the data cannot be made available publicly (e.g., because it is yet to be formally published), or if there are some other confidentiality issues with submitting the data, speak with your course director and supervisor, and include a clear statement about why the data cannot be made available under the same Code and Data Availability header. Note that most data repositories allow timed embargos on data (e.g., Zenodo; see http://about.zenodo.org/policies/).
\chapter{Acknowledgements}
Thank you to my supervisors, Ryan, James and Tom. Thank you to all of the people that took the time to share their data with me.

\bibliographystyle{apalike}
\bibliography{bibliography}
\addcontentsline{toc}{chapter}{Bibliography}

\chapter{Supplementary Material}

\section{Datasets}

\end{document}