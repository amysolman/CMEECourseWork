\chapter{Introduction}

%What are SARs and why are they important?
%Why is the theory important? 
The species--area relationship (SAR) is one of the oldest fundamental ecological laws \cite{GooriahLeanaD2019Sedt}. Positive SARs, where number of species increases with sampling area, has been observed for a broad range of faunal \cite{ricklefs1999roles} \cite{lomolino1982species} \cite{eadie1986lakes} and floral groups \cite{zacharias1990species} \cite{price2011phylogenetic}. The ubiquitous nature of positive SARs has been used to inform conservation practises in natural \cite{haila2002conceptual} \cite{samson1980island} and urban environments \cite{davis1978urban}. Whilst a large number of studies have examined macroorganism SARs, little is known about the spatial scaling of microbial biodiversity. \\

%Why is it important?
{\texorpdfstring
\noindent UUnderstanding the factors that regulate microbial community structure is important as they play a vital role in biogeochemical cycling and ecosystem functioning \cite{griffiths2011bacterial}. Little is known about below-ground regulators of biodiversity as few terrestrial environments present insular habitats for microbial community dynamics to be easily studied. Microorganisms also play a metabolically active role in polar regions previously believed to be abiotic \cite{stibal2020glacial}. Rapid climate change is leading to exposure of soils dominant in high-latitude carbon \cite{bradley2017microbial}. Understanding the mechanisms that drive microbial colonisation of polar environments can help produce accurate climate models \cite{malard2018microbial}.} \\

%History of Microbial SARs
\noindent Debate around the applicability of SARs to microbial systems stems from the assumption that they are limited only by niche filtering, as articulated by Bass-Becking: `\textit{Everything is everywhere}, but, \textit{the environment selects}' \cite{baas1934geobiologie}. This classic tenet of microbiology assumes that the abundance, short generation times and small size of microorganisms gives them an almost cosmopolitan distribution \cite{GreenJessica2006Ssom}. High abundances increase the probability of transport between environments via an accidental vector. Small size increases the likelihood of passive transport via air or water,\cite{GreenJessica2006Ssom}. Uninhibited dispersal may also be facilitated by dormancy as a biogeographic response \cite{LoceyKennethJ2010Stbw}. \\

%The species-area relationship power-law model
\noindent One of the most commonly used tools in biogeography is the power-law model:

\begin{equation}
S=cA^{z}
\end{equation}\\

\noindent Where \textit{S} is species richness as a function of area (\textit{A}), \textit{c} is a constant specific to that taxa/habitat and the \textit{z} exponent is the slope of the line associating area and species richness \cite{darcy2018island}. \textit{z} typically falls in the range of 0.1 to 0.3 for continuous habitats and 0.25 to 0.35 for insular habitats \cite{GreenJessica2006Ssom}. Microbial \textit{z} values are typically well below those seen in macro-organisms (\textit{z} < 0.1), supporting the idea of cosmopolitan distribution \cite{GreenJessica2006Ssom}. \\

%Problems with Microbial SARs
%Distribution maps define the annual or seasonal spatial distributions of functional groups and life stages, for simulating spatial patterns of predator-prey interactions
\noindent One of the limitations for microbial biogeography has been in quantifying taxa, given that many cannot be accurately identified using morphological techniques \cite{GreenJessica2006Ssom}. Microbial biogeography is usually concerned with taxa-area relationships (TARs), rather than SARs as microbial diversity is quantified in operational taxonomic groups (OTUs). With recent advances in molecular approaches such as single-celled sequencing, the genomes of previously uncultivated bacterial taxa are filling in the microbial phylogenetic tree \cite{lasken2014recent}. Limited data on temporal and spatial microbial distributions has led to a lack of detailed distribution maps, inhibiting taxa richness estimations and artificially lowering \textit{z} values \cite{GreenJessica2006Ssom}.\\


\noindent The mechanisms driving island SARs have been of particular interest to ecologists since the 1800s \cite{macdonald2018theory}. Islands are important paradigms for fragmented habitats and larger geographic regions \cite{simberloff1974equilibrium}. Their insular nature allows for ecological patterns to be investigated in a simplified and relatively closed system. \\

%The general theory of island biogeography
%What is the theory about?

\begin{figure}[htp]

\centering
\includegraphics[width=.5\textwidth]{ColonisationDynamicEquilibrium.pdf}\hfill

\caption{Colonisation-Extinction Dynamic Equilibrium}
\label{fig:figure1}

\end{figure}

\noindent MacArthur and Wilson's Theory of Island Biogeography \cite{MacArthurRobertH1967Ttoi} is one of the most widely accepted island SAR theories, explaining the maintenance of biodiversity on islands through stochastic processes of colonisation and extinction. The rates of these processes are determined by island area and isolation. Islands that are nearer to source populations will experience a higher rate of immigration. This in turn can produce a rescue effect leading to decreased extinction risk \cite{brown1977turnover}. Larger islands receive more immigrants as species actively target larger habitats with more resources, or will be more likely to immigrate randomly due to island size. A larger population is also less susceptible to inbreeding depressions and random extinction \cite{macdonald2018theory}. This results in higher species richness at the point of balance between immigration and extinction rates (i.e. the colonisation-extinction dynamic equilibrium, Figure 1.1) for larger, less isolated islands. \\

%What other mechanisms can drive patterns of biogeography?
\noindent It has been suggested the significance of area has been overplayed, to the exclusion of interspecific relationships, biotic and abiotic factors \cite{abbott1974numbers}. Empirical evidence indicates smaller islands do not always follow the positive SAR pattern  \cite{triantis2006re} \cite{sfenthourakis2009habitat}. MacArthur and Wilson noted that archipelagos showed unusual SARs, with smaller island species-richness varying independently of size \cite{MacArthurRobertH1967Ttoi}. This exception to MacArthur and Wilson's ecological law has been dubbed the small-island effect (SIE). \\

%Explainations of the SIE
\noindent Several hypotheses have been offered to explain the SIE. The 'subsidised island biogeography' hypothesis suggests that smaller islands have a greater edge to interior ratio, thus receive a greater amount of nutrients per unit area \cite{barrett2003small} \cite{anderson2001subsidized}. Secondly, extinction rates on islands may operate independently of area due to environmental instability and high temporal turnover \cite{MacArthurRobertH1967Ttoi}. Thirdly, the 'habitat hypothesis' suggests that diversity is limited on smaller islands, compared to larger islands \cite{triantis2008evolutionary}. However, the environmental instability and habitat hypotheses contradict empirical data that indicate small islands have unusually high numbers of species. The Habitat--Diversity Hypothesis addresses the SIE by stating that as observation area increases we encounter a greater range of habitats \cite{EdwardF.Connor1979TSaB}. Species richness should increase with habitat diversity, which varies independently of area \cite{macdonald2018theory}.  \\

%Chisholm et al model and research
\begin{figure}[htp]

\centering
\includegraphics[width=.5\textwidth]{LowImIslands.png}\hfill

\caption{A graphical representation of a simulation (using the Classic Model, see Methods) of three islands of varying size, with the same number of niches (\textit{K}=4) and \textbf{low immigration rate} (\textit{m} = 0.03). Each of the three main squares represents an island. Each smaller square represents an individual niche. Each unique colour patch within a niche represents a unique species. The smallest island has one individual per niche, the medium size island has four individuals per niche and the largest island has nine individuals per niche. Species richness on the smallest island is \textbf{4}, on the medium island is \textbf{5} and the large island is \textbf{6}}
\label{fig:figure2}

\end{figure}

\begin{figure}[htp]

\centering
\includegraphics[width=.5\textwidth]{HighImIslands.png}\hfill


\caption{A graphical representation of a simulation (using the Classic Model, see Methods) of three islands of varying size, with the same number of niches (\textit{K}=4) and \textbf{high immigration rate} (\textit{m} = 0.9). Each of the three main squares represents an island. Each smaller square represents an individual niche. Each unique colour patch within a niche represents a unique species. The smallest island has one individual per niche, the medium size island has four individuals per niche and the largest island has nine individuals per niche. Species richness on the smallest island is \textbf{4}, on the medium island is \textbf{15} and the large island is \textbf{33}}
\label{fig:figure3}

\end{figure}

\noindent Chisholm \textit{et al.,} (2016) explain both deterministic and stochastic SARs in a unified theory. They posit this pattern is due to transition from a niche-structured regime on smaller islands, to colonisation-extinction regime on larger islands. The niche-structured regime is characteristic of deterministic theories like the Habitat--Diversity Hypothesis, where habitat structure and intra- and interspecific interactions determine species richness \cite{chase2011disentangling}. The colonisation-extinction regime is characteristic of stochastic mechanisms such as the Theory of Island Biogeography and ecological neutral theory, where richness is dictated by random colonisation and extinction events, as well as ecological drift \cite{hubbell2001unified}. Chisholm \textit{et al.,} hypothesise that species richness on all islands is maintained by these two mechanisms, where niche diversity increases slowly with area and immigration rate increases quickly. Smaller islands are constrained by niche-structured regimes, until a critical area threshold where species richness is constrained by immigration. Figures 1.2 and 1.3 show the effect of immigration rate and area on species richness, where each `island' is made of four niches and the different coloured patches inside each niche represent a species unique to that niche. Small islands where each niche supports a small number of individuals will be constrained by those niches and species richness will not vary considerably with immigration rate. Larger island species richness varies considerably with immigration rate as they are less constrained by the number of niches and diversity is dictated by immigration and extinction. Chisholm \textit{et al.,} developed a parsimonious mechanistic model to test their hypotheses, and applied it to 100 archipelago datasets. Their results supported the prediction that critical area will be lower for species with higher motility and less isolated habitats. \\

%what do we know about the mechanisms driving microbial TARs? Deterministic niche regimes vs stochastic neutral regimes
\noindent Previous research indicates that microbial TARs may be controlled by either deterministic or stochastic processes \cite{StegenJamesC2012Sada}. Phylogenetic analysis of subsurface microbial communities showed related taxa utilised similar habitats, illustrating that environmental filtering determined community composition \cite{StegenJamesC2012Sada}. Niche filtering has a greater influence in spatially and temporally diverse environments and varies with community functionality \cite{StegenJamesC2012Sada} \cite{CarusoTancredi2011Sadp}. \\

\noindent Whilst both stochastic and deterministic processes affect microbial communities, few studies discuss the transition of mechanisms across a spatial scale. An investigation of phytoplankton TARs in water bodies indicated that for the smallest spatial scales, niche structure determine OTU richness, before transitioning to an immigration dominated regime \cite{varbiro2017functional}. The SIE has also been seen in benthic diatoms where stochastic variation in OTU richness is a function of small habitat instability \cite{bolgovics2016species}. \\

%Aquatic Microbial TARs
\noindent Aquatic habitats are some of the most studied in microbial biogeography due to the availability of insular water bodies and their range of spatial scales. An investigation into bacterial diversity in aquatic tree holes found a \textit{z} value comparable to macro-organisms (\textit{z} = 0.26) \cite{bell2005larger}. Antarctic cryoconite holes have also exhibited positive TARs on glaciers where dispersal is limitated \cite{darcy2018island}. Positive TARs have been reported for habitats as diverse as lakes \cite{battes2019species}, membrane bioreactors \cite{van2006bacterial} \cite{van2005island} and vertebrate bodies \cite{godon2016vertebrate}.\\

%Terrestrial Microbial TARs
\noindent Previous investigations into ectomycorrhizal fungi communities within 'tree island' root systems showed that total OTU richness increased significantly with size, although distance effects vary \cite{glassman2017theory} \cite{peay2007strong}. Bacterial and fungal diversity has been positively correlated with area but via different mechanisms \cite{li2020island}. Country and continent-scale patterns of pathogen diversity have also been shown to be a function of area and isolation \cite{jean2016equilibrium} \cite{cashdan2014biogeography}. In both terrestrial and aquatic systems microbial communities exhibit significant TARs. The varying mechanisms underlying these TARs warrant further investigation. \\

\noindent In this project I apply three modified versions of the model presented by Chisholm \textit{et al.,}: \\

\noindent {$\cdot$ \textbf{Classic Model}: Where per capita immigration rate is proportional to area (e.g. in the case of aerial and directed dispersal species immigrating into a two-dimensional habitat)} \\
\noindent {$\cdot$ \textbf{Perimeter Model}: Where per capita immigration is proportional to perimeter (e.g. in the case of water or soil dispersed species immigrating into a two-dimensional habitat) } \\
\noindent {$\cdot$ \textbf{Depth Model}: Where per capita immigration rate is proportional to depth (e.g. in the case of species dispersing into a volume via a surface portal (three-dimensional habitat)) }\\

\noindent These models are applied to bacterial, archaeal and micro-eukaryote insular spatial data with the aim of testing whether there is a biphasic microbial TAR, as well as investigating the impact of habitat type and taxonomic group on critical area of transition between the deterministic and stochastic regimes. 