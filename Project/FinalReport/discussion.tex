\chapter{Discussion}

%DONE
I have presented three variations of the Chisholm model \cite{chisholm2016maintenance} that take into account varying habitats and immigration routes and have successfully fit all three to microbial TAR data. The relatively equal success of the three model variations (Classic, Depth, Perimeter) suggests that immigration route is not a significant factor in defining microbial TARs (see Supplementary Materials, Table 7.5).  Microorgansims can cross oceanic, glacial and global barriers through a variety of dispersal mechanisms \cite{rosselli2015microbial} \cite{darcy2018island} \cite{kleinteich2017pole}. Microbial OTUs utilise a variety of immigration routes when entering a new environment. No significant accuracy was lost in assessing three-dimensional habitats using the two-dimensional models (Classic, Perimeter), suggesting habitat depth did not affect OTU richness as strongly as area, where area acts as immigration portal into the three-dimensional habitat. Algae and bacteria have shown negative correlations between OTU richness and depth \cite{battes2019species} \cite{turner2017microbial}. Nutrient-poor, low-energy environments may accompany increasing habitat depth. The immigration portal may be characterised by a nutrient-rich, high-energy stratification that is a more potent predictor of OTU richness than depth. \\

%%%%talk about the power law model
%DONE
\noindent The phenomenological power-law model did not perform significantly better than the mechanistic Classic, Depth and Perimeter models, according to the AIC measure of parsimony, relative to model fit (see Supplementary Materials, Table 7.4). This indicates that the model parameters ($\theta$, \textit{m\textsubscript{0}}, \textit{K}, $\rho$) were useful in describing TARs, supporting the hypothesis that OTU richness is influenced by the parameters, rather than being a constant power of area. The mean slope of positive TARs across these datasets was comparable to macroorganisms (\textit{z}=0.16) and higher than those previously reported for microbial taxa \cite{rosenzweig1995species} \cite{green2004spatial} (see Supplementary Materials, Table 7.3). These observations show habitat area has a relatively strong influence on OTU richness.  \\

%DONE
\noindent Successfully fitted datasets exhibit both the classic MacArthur and Wilson \cite{MacArthurRobertH1967Ttoi} biogeographic pattern and the SIE. The results demonstrate that some microbial communities are constrained by niche-structured regimes at smaller areas where immigration is low, before transitioning to colonisation-extinction balance regimes at larger areas where immigration is high. This lends support to the theory that microbial species are not ubiquitous and unlimited in dispersal, that they can be limited by habitat heterogeneity, resource availability and dispersal barriers, but this is not a ubiquitous pattern with over 50\% of the datasets failing to be fit by the model. \\

%DONE
\noindent Many datasets with positive TARs could not be successfully fit with the power-law, Classic, Depth or Perimeter models (see Supplementary Materials, Figure 7.2). Despite positive \textit{z} values, confidence intervals included zero and were not statistically significant. Stochastic variation between data points inhibited identification of significant TARs (see Supplementary Materials, Figure 7.2 a \& b). The majority of failed fits were aquatic habitats and may be due to uncertainty in the spatial sample regime of a heterogenous habitat. In order to elucidate these patterns it might be useful to take a stratified approach. Some failed fits had too few data points in comparison to the number of model parameters, producing low adjusted R\textsuperscript{2} values (see Supplementary Materials, Figure 7.2 b). Microbial TARs may also be undetected due to the disparity between sample and community sizes meaning rare taxa are missed \cite{woodcock2006taxa}.\\ 

%DONE
\noindent This project is, to the best of my knowledge, the first attempt to apply a biphasic mechanistic TAR model to microbial data. The model demonstrates that when niche diversity increases slowly or remains constant and immigration increases quickly with area, a biphasic TAR is produced. At an \textit{A\textsubscript{crit}} specific to that habitat and taxonomic group, the TAR will transition from deterministic to a stochastic mechanisms. I hypothesised that \textit{A\textsubscript{crit}} would be lower where immigration is higher (i.e. for more motile OTUs and less isolated habitats). My analysis indicated that taxonomic group was significant in predicting \textit{A\textsubscript{crit}}, while habitat type was not. Taxonomic group is significant in predicting \textit{A\textsubscript{crit}} as taxa are constrained (or liberated) by their life stages (activity and dormancy) and dispersal methods. According to this data isolated habitats present no significant dispersal barriers to microorganisms as a whole, although their relative accessibility varies between taxonomic groups. Overall, after removing the outlying datapoint and removing habitat type as an explanatory variable, the model accounts for nearly half of the variation in log \textit{A\textsubscript{crit}} using broad taxonomic groups.\\

%DONE
\noindent My analysis indicated that pathogenic OTUs had overwhelmingly higher \textit{A\textsubscript{crit}} values (mean 9.43 x 10\textsuperscript{30} cm\textsuperscript{2}), suggesting they are more constrained by resource availability and dispersal barriers. This may be due to their dependence on host species, although this will be directly related to the motility and sociability of their hosts. The two datasets used in this study quantify human pathogen richness on `true' islands \cite{jean2016equilibrium}. Human pathogen OTU richness is negatively correlated with disease control efforts \cite{dunn2010global}. I suggest that global mitigation strategies such as behavioural change, medicine and vaccination mean pathogens face considerable dispersal barriers that limit immigration and constrain them to niche-structured spatial regimes over larger areas  \cite{nicolaides2020hand} . \\

%DONE
\noindent Bacterial OTUs exhibited the lowest mean \textit{A\textsubscript{crit}} value (3.02 x 10\textsuperscript{3} cm\textsuperscript{2}). The small size of bacteria allows them to dispersal more freely than size-limited macroorganisms \cite{martiny2006microbial}. They may also overcome dispersal limitation through dormancy and enormous population sizes \cite{LoceyKennethJ2010Stbw} \cite{fenchel2004ubiquity}. Bacteria have a variety of ecological traits that allow them to move freely and access isolated habitats, thus they transition to stochastic TAR mechanisms at lower areas. \\
 
 %DONE
\noindent Fungi also showed low \textit{A\textsubscript{crit}} values (mean 1.72 x 10\textsuperscript{16} cm\textsuperscript{2}). Mycorrhizal fungi that share beneficial associations with plant roots, have large spores that immigration slowly through soil \cite{bueno2019arbuscular}, however, the close proximity of potential host plants might mitigate low fungal motility. For other fungal groups, long distance spore dispersal is facilitated by meteorological, biotic and anthropogenic vectors \cite{golan2017long}. Fungal sporulation allows taxa to overcome local and regional barriers, thus contributing to the low \textit{A\textsubscript{crit}} values seen in these datasets. \\

%DONE
\noindent Algae (mean 2.56 x 10\textsuperscript{19} cm\textsuperscript{2}) and protozoa (mean 7.43 x 10\textsuperscript{23} cm\textsuperscript{2}) exhibited similar broad, midrange \textit{A\textsubscript{crit}} values which may be due issues of sampling in spatially heterogenous aquatic environments. Issues of taxonomic classification, particularly for protists, may contribute to varying estimations of diversity \cite{foissner2006biogeography}. Whilst it seems algae and protists transition from deterministic to stochastic mechanisms in the midrange of areas, further investigation is needed to discern a true pattern within the wide range of \textit{A\textsubscript{crit}} values estimated. The multiple regression model coefficients (Table 3.7) broadly confirm the overall taxonomic results. \\

%DONE
\noindent As the multiple regression analysis showed that habitat type was non-significant in predicting \textit{A\textsubscript{crit}} I cannot assess the relative isolation of habitats or how they may affect \textit{A\textsubscript{crit}}. The non-significance of habitat type, despite marked differences in the mean \textit{A\textsubscript{crit}} is due to the large, overlapping estimate ranges. It is interesting however to look at mean \textit{A\textsubscript{crit}} values for each habitat, as a sign post towards what may be found with a more comprehensive dataset. Terrestrial habitats show the highest mean \textit{A\textsubscript{crit}} values (2.36 x 10\textsuperscript{30} cm\textsuperscript{2}). This may be due to immigration via an accidental vector being limited to aerial species that can reach land islands. Passive immigration to land islands relies on stochastic success which may limit dispersal, although fungal and bacteria OTU richness on land islands has been shown be unaffected by isolation \cite{li2020island}. \\

%DONE
\noindent Lacustrine habitats exhibited low mean \textit{A\textsubscript{crit}} values (1.28 x 10\textsuperscript{19} cm\textsuperscript{2}) suggesting immigration to these habitats is high. Aquatic taxa utilise a variety of dispersal mechanisms between habitats, including dispersal via insects and waterfowl \cite{stewart1966dispersal}. It may be easier for microbial OTUs to colonise inland lacustrine environments where animal activity increases the probability of vector transport. There may be higher rates of passive transport to lacustrine environments due to the interconnectivity of rivers and streams that empty into watershed areas, filling lakes and ponds.  

%DONE
\noindent Plant habitats also have low mean \textit{A\textsubscript{crit}} values (2.01 x 10\textsuperscript{4} cm\textsuperscript{2}). For many symbiotic plant-microbe species relationships, plant seeds are already inoculated with associated microbial taxa on dispersal \cite{ho2017plant}. Thus dispersal barriers between plant and microbes are removed, contributing to low \textit{A\textsubscript{crit}} values. Many plant communities are comprised of the same species in close proximity, providing ready access to source populations and increasing immigration. \\

%DONE
\noindent Four of the six best-fit datasets were for bacteria in machine habitats \cite{van2006bacterial} \cite{van2005island}. It may be the strong TAR is a function of isolation, relative to natural habitats. When constrained by immigration more prominent and easily quantifiable TAR patterns arise. The model fitting process supports this by estimating extremely low immigration rates for machine habitats. Despite this, machine habitats had the lowest mean \textit{A\textsubscript{crit}} (6.56 x 10\textsuperscript{3} cm\textsuperscript{2}). \textit{A\textsubscript{crit}} is not only affected by immigration as in my primary hypothesis, but can also be affected \textit{K}, $\rho$ and $\theta$. In the fitted model the low \textit{A\textsubscript{crit}} for machine habitats in spite of their isolation is caused by the low \textit{K} values of homogenous, man-made environments, a characteristic of these unusual habitats that warrants further investigation. \\

%DONE
\noindent The reason for the lack of successful fittings for riverine habitats is due to the low number of data points that forced low adjusted R\textsuperscript{2} scores, despite initial good fits (see Table 7.1, Supplementary Material). Only one dataset included archaeal TARs and no significant relationship between area and OTU richness was found. This is likely due to the importance of environmental filtering for extremophile OTUs in soda lakes \cite{LanzenAnders2013SPaE}. It is interesting to note other datasets removed from the fitting process due to a lack of positive TARs. These included, fungi in the Antarctic cryoconite holes of two glaciers where extreme biomass influx negated observable TARs \cite{darcy2018island}. Inappropriate diversity metrics and spatial scaling may have led to undetectable root-symbiotic fungi TARs \cite{davison2018microbial}. Fungi OTU richness did not increase with area on submerged leaves due to a lack of energy increase with corresponding area as expressed by the species-energy theory \cite{FeinsteinLarryM2012Tran}  \cite{wright1983species}. TARs may not have arisen in protozoan communities due to a failure to reach equilibrium (dataset 55) \cite{henebry1980effect}. It is clear that the factors affecting microbial TARs are diverse and each habitat/taxa pairing may require unique assessment. \\ 

%DONE
%What was the anomalous result? Why was it anomalous?
\noindent The anomalous result removed from analysis concerned pathogenic bacterial OTU richness on `true' geographic islands \cite{jean2016equilibrium}. The model was a poor fit to the data (R\textsuperscript{2}=0.23, adjusted R\textsuperscript{2}=0.18) and it's likely the error associated with estimating density for pathogenic bacteria over such large geographic scales lead to poor estimations of the remaining parameters and excessive \textit{A\textsubscript{crit}}.   \\

%DONE
%What are the caveats that apply to this study? (Leave out caveats that apply to all studies.)
\noindent Whilst the data indicate that habitat type is non-significant in predicting \textit{A\textsubscript{crit}}, the large range in \textit{A\textsubscript{crit}} values suggests there many be too few data points to discern a significant pattern. I also encountered challenges when trying to compare studies that used a variety of quantification techniques. Microbial OTUs inhabit three-dimensional habitats and whilst steps have been taken to account for this there is more work to be done to incorporating this fully. In an extension of this model I would consider each stratification of a habitat separately, to account for spatial heterogeneity. Volume has been shown to be more accurate in quantifying microbial TARs \cite{van2006bacterial}. It would be useful to further modify the model to explicitly incorporate volume and \textit{V\textsubscript{crit}} across datasets, as nearly all of them concern habitats within a volume even though often only area data is available. Here I have used area with a depth metric (Depth Model) which suggests the habitat maintains the same area for the full depth, whereas natural habitats rarely take this shape and this reduces the accuracy of my results. \\

%DONE
\noindent Another issue I encountered was $\rho$ estimations as direct counts are rarely given for microbial OTUs. Estimations were made in various ways, using gene sequence numbers or proxy papers. It would be beneficial to develop more robust methods for estimating $\rho$ as data from proxy papers introduces error. A broad scale experiment to quantify microbial TARs in a laboratory, where data specific to the needs of these models could be collected would provide a more vigorous assessment of the models applicability to microbial TARs. \\

%DONE
\noindent When validating my fitting procedures there was small but significant error between true and estimated parameters. Parameter ranges of speciation rate, \textit{m\textsubscript{0}} and \textit{K} were not inferred from microbial ecological theory, but were selected for ease of computation. A more thorough exploration of the parameter space, with ecologically relevant parameter ranges, to further validate the fitting procedure and the areas of parameter space where there may be errors in fitting would be desirable in further work. \\

%DONE
\noindent There remains to be a thorough synthesis between biogeography and microbial ecology. Here I have gone some way to evaluate the influence of immigration on microbial TARs, however more work is needed to examine dispersal barriers. Dormancy is a widespread microbial response that may allow OTUs to overcome dispersal barriers and increase immigration to new habitats. However, it is a slow, passive process that will not necessarily lead the individual to a viable habitat \cite{LoceyKennethJ2010Stbw}. To fully elucidate the interplay of microbial ecology and biogeographic patterns, work is needed to incorporate dormancy as a biogeographic response.  \\         

%What are the implications of this work?
%DONE
\noindent An implication of this work is that if we can identify the niches within a habitat and the taxonomic groups that tend to occupy those niches, we may be able to better predict OTU richness at a range of spatial scales. This presents a complex challenge that requires the integration of environmental variables and habitat stratification. If these challenges could be overcome, it would be a useful tool in predicting colonisation of new habitats such as soil exposed by glacier retreat, helping us model the biogeochemical processes of colonisation. \\      

%A nice wrap-up, emphasising how this study in this system is of interest to people who work on other things, or other systems.
\noindent This study has demonstrated that microbial communities in isolated `island' habitats can be subject to niche-structured regimes, before a critical area of transition to colonisation--extinction regimes. I have shown that taxonomic group is significant in predicting critical area, but habitat type is not. The overwhelming number, complexity and importance of microbial life illustrates the need to understand their biogeographic patterns. I hope that my study will lead to further research into deterministic and stochastic mechanisms in microbial biogeography, as well as the importance of taxonomic group on the relative influence of these processes. The synthesis of microbial ecology and biogeography will be of increasing interest as climate change alters habitats, extending microbial ranges and leading to climate feedback loops. Microbial biogeography is an essential area of study in our global challenge to predict and mitigate the impacts of climate change. Everything is \textit{not} everywhere, and everything is changing.             