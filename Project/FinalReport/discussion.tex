\chapter{Discussion}

%Discussion. This should attempt to tie together the results, what they indicate in a broader context, the extent to which the original aims have been satisfied and what future work is suggested. Return to and address the ideas raised in the introduction. In particular, think about:
%What’s the main thing we know now that we didn’t know before?
%What’s the chain of logic and results that means we know it?
%How does this affect our -- and other scientists’ -- view of the world? What are the implications?
%What are the implications of the intermediate steps in the chain towards the main thing?

%Plan:
%What have I essentially demonstrated? 

%DONE
I have presented three variations of the Chisholm model \cite{chisholm2016maintenance} that take into account varying habitats and immigration routes and have successfully fit all three to microbial TAR data. The relatively equal success of the three model variations (Classic, Depth, Perimeter) suggests that immigration route is not a significant factor in defining microbial TARs (see Supplementary Materials, Table 7.5).  Microorgansims can cross oceanic barriers via airborne dust particles \cite{rosselli2015microbial}, enter glacial habitats via stream deposition \cite{darcy2018island} and reach pole-to-pole via airborne, animal vector and anthropogenic mediated dispersal \cite{kleinteich2017pole}. Microbial OTUs likely utilise a variety of immigration routes when entering a new environment. No significant accuracy was lost in assessing three-dimensional habitats using the two-dimensional models (Classic, Perimeter), suggesting habitat depth did not affect OTU richness as strongly as area, where area acts as immigration portal into the three-dimensional habitat. Algae and bacteria have shown negative correlations between OTU richness and depth \cite{battes2019species} \cite{turner2017microbial}. These patterns emerge as increasing habitat depth often accompanies nutrient-poor, low-energy environments. The immigration portal may be characterised by a nutrient-rich, high-energy stratification that is a more potent predictor of OTU richness than depth. \\

%%%%talk about the power law model
%DONE
\noindent The phenomenological power-law model did not perform significantly better than the mechanistic Classic, Depth and Perimeter models, according to the AIC measure of parsimony, relative to model fit (see Supplementary Materials, Table 7.4). This indicates that the model parameters ($\theta$, \textit{m\textsubscript{0}}, \textit{K}, $\rho$) were useful in fully describing the shape of the observed TARs, thus supporting the hypothesis that OTU richness is influenced by the parameters, rather than being simply a constant power of the area. The mean slope of positive TARs across these datasets was comparable to macroorganisms (\textit{z}=0.16) and higher than those previously reported for microbial taxa \cite{rosenzweig1995species} \cite{green2004spatial} (see Supplementary Materials, Table 7.3). These observations show habitat area has a relatively strong influence on OTU richness. Isolated habitats may provide the stability needed for microbial taxa to reach equilibrium and for stronger TARs to arise, in contrast to turbulent continuous habitats \cite{bell2005larger}.    \\

%DONE
\noindent The successfully fitted datasets exhibit both the classic MacArthur and Wilson \cite{MacArthurRobertH1967Ttoi} biogeographic pattern of increasing OTU richness with area and the small island effect of OTU richness varying independently of area. The results demonstrate that some microbial communities are constrained by niche-structured regimes at smaller areas where immigration is low, before transitioning to colonisation-extinction balance regimes at larger areas where immigration is high. This lends support to the theory that microbial species are not ubiquitous and unlimited in dispersal, that they can be limited by habitat heterogeneity, resource availability and dispersal barriers, but this is not a ubiquitous pattern with over 50\% of the datasets failing to be fit by the model. \\

%DONE
\noindent Many datasets with positive TARs (as indicated by Spearman's rank correlation coefficients) could not be successfully fit with the power-law, Classic, Depth or Perimeter models (see Supplementary Materials, Figure 7.2). Despite positive \textit{z} values for these datasets, confidence intervals included zero and therefore were not statistically significant. Stochastic variation between data points inhibited the models from discerning significant TARs (see Supplementary Materials, Figure 7.2 a \& b). The majority of failed fits were aquatic habitats and may be due to uncertainty in the spatial sample regime of a heterogenous habitat. In order to elucidate the the spatial patterns within these habitats, it might be useful to take a stratified approach. Some of the failed fits had too few data points in comparison to the parameters of the model, producing low adjusted R\textsuperscript{2} values (see Supplementary Materials, Figure 7.2 b). Microbial TARs may also be undetected due to the disparity between sample and community sizes meaning rare taxa are missed \cite{woodcock2006taxa}.\\ 

%\noindent The majority of failed fittings were for aquatic habitats. For my successful fittings, aquatic groups such as algae and protozoa show wide ranging \textit{A\textsubscript{crit}} results suggesting their is uncertainty caused by spatial sampling regime in a spatially heterogenous environment.  Due to the data available and for computational simplicity, OTUs were considered evenly distributed throughout the water column, however, OTUs are subject to spatial heterogenous distributions throughout a three-dimensional habitat \cite{barberan2011euxinic}. In order to clearly elucidate TARs in any three-dimensional environment it may be necessary to quantifying OTU richness in each stratification separately in addition to analysing the habitat as a whole.   

%DONE
\noindent This project is, to the best of my knowledge, the first attempt to apply a biphasic mechanistic TAR model to microbial data. The model demonstrates that when niche diversity increases slowly or remains constant and immigration increases quickly with area, a biphasic TAR is produced. At an \textit{A\textsubscript{crit}} specific to that habitat and taxonomic group, the TAR will transition from deterministic to a stochastic mechanisms. I hypothesised that \textit{A\textsubscript{crit}} would be lower where immigration is higher (i.e. for more motile OTUs and less isolated habitats). My analysis indicated that taxonomic group was significant in predicting \textit{A\textsubscript{crit}}, while habitat type was not. It is likely that taxonomic group is significant in predicting \textit{A\textsubscript{crit}} as taxa are constrained (or liberated) by their own range of life cycles (activity and dormancy) and dispersal methods (sporulation, meteorological, biotic, anthropogenic, passive). According to this data isolated habitats present no significant dispersal barriers to microorganisms as a whole, although their relative accessibility varies between taxonomic groups.\\

%DONE
\noindent My analysis indicated that pathogenic OTUs had overwhelmingly higher \textit{A\textsubscript{crit}} values (mean 9.43 x 10\textsuperscript{30} cm\textsuperscript{2}), thus they are more constrained by resource availability and dispersal barriers. I suggest this is due to their dependence on host species, although this will be directly related to the motility and sociability of their hosts. The two datasets used in this study quantify human pathogen richness on 'true' islands \cite{jean2016equilibrium}. Human pathogen OTU richness is negatively correlated with disease control efforts \cite{dunn2010global}. I suggest that global mitigation strategies such as behavioural change, medicine and vaccination \cite{nicolaides2020hand} mean pathogens face considerable dispersal barriers that limit immigration and constrain them to niche-structured spatial regimes over larger areas. \\

%DONE
\noindent Bacterial OTUs exhibited the lowest mean \textit{A\textsubscript{crit}} value (3.02 x 10\textsuperscript{3} cm\textsuperscript{2}). The small size of bacteria allows them to dispersal more freely than size-limited macroorganisms \cite{martiny2006microbial}. They may also overcome dispersal limitation through dormancy as a biogeographical response and as a consequence of enormous population sizes \cite{LoceyKennethJ2010Stbw} \cite{fenchel2004ubiquity}. Bacteria have a variety of ecological traits that allow them to move freely and access isolated habitats, thus they transition to stochastic TAR mechanisms at lower areas. \\
 
 %DONE
\noindent Fungi also showed low \textit{A\textsubscript{crit}} values (mean 1.72 x 10\textsuperscript{16} cm\textsuperscript{2}). Mycorrhizal fungi, where there are beneficial associations with plant roots, have large spores that immigration slowly through soil \cite{bueno2019arbuscular}, however, the close proximity of potential host plants might mitigate low fungal motility. For other fungal groups, long distance spore dispersal is facilitated by meteorological, biotic and anthropogenic vectors \cite{golan2017long}. Fungal sporulation allows taxa to overcome local and regional barriers, thus contributing to the low \textit{A\textsubscript{crit}} values seen in these datasets. \\

%DONE
\noindent Algae (mean 2.56 x 10\textsuperscript{19} cm\textsuperscript{2}) and protozoa (mean 7.43 x 10\textsuperscript{23} cm\textsuperscript{2}) exhibited similar midrange \textit{A\textsubscript{crit}} values. The broad range of \textit{A\textsubscript{crit}} values for these taxonomic groups may be due to the issue of spatial sampling regime in spatially heterogenous aquatic environments. Issues of taxonomic classification, particularly for protists may contribute to varying estimations of diversity \cite{foissner2006biogeography}. Whilst seems that algae and protists transition from deterministic to stochastic mechanisms of spatial scaling in the midrange of areas, further investigation is needed to discern a true pattern within the wide range of  \textit{A\textsubscript{crit}} values estimated. The multiple regression model coefficients (Table 3.7) broadly confirm the overall taxonomic results. \\

%DONE
\noindent As the multiple regression analysis showed that habitat type was non-significant in predicting \textit{A\textsubscript{crit}} I cannot assess the relative isolation of habitats or how they may affect \textit{A\textsubscript{crit}}. It is interesting however to look at the mean \textit{A\textsubscript{crit}} values for each habitat type, as a sign post towards what may be found with a more comprehensive dataset. Terrestrial habitats show the highest mean \textit{A\textsubscript{crit}} values (2.36 x 10\textsuperscript{30} cm\textsuperscript{2}). This may be due to immigration via an accidental vector being limited to aerial species that can reach the land island. Passive immigration by water or air to land islands relies on stochastic success which may limit dispersal, although fungal and bacteria OTU richness has been shown be unaffected by isolation \cite{li2020island}. \\

%DONE
\noindent Lacustrine habitats exhibited low mean \textit{A\textsubscript{crit}} values (1.28 x 10\textsuperscript{19} cm\textsuperscript{2}) suggesting immigration to these habitats is high. Aquatic taxa such as algae and protozoa utilise a variety of dispersal mechanisms between habitats, including dispersal via insects and waterfowl \cite{stewart1966dispersal}. It may be easier for microbial OTUs to colonise inland lacustrine environments where animal activity increases the probability of transport via an accidental vector. Passive transport to lacustrine environments may have a greater success rate than terrestrial islands due to the interconnectivity of rivers and streams that empty into watershed areas, filling lakes and ponds.  

%DONE
\noindent Plant habitats also have low mean \textit{A\textsubscript{crit}} values (2.01 x 10\textsuperscript{4} cm\textsuperscript{2}). For many symbiotic plant-microbe species relationships, plant seeds are already inoculated with associated microbial taxa on dispersal \cite{ho2017plant}. Thus dispersal barriers between plant and microbes are removed, contributing to low \textit{A\textsubscript{crit}} values. Many plant communities are comprised of the same species in close proximity, providing ready access to source populations and increasing immigration. 

%DONE
Four of the six best-fit datasets were for bacteria in machine habitats (membrane bioreactors and metal cutting machine sump tanks) \cite{van2006bacterial} \cite{van2005island}. It may be that the strong TAR found in these environments is a function of their isolation, relative to natural habitats. Despite the large numbers, rapid asexual reproduction and resilience to extinction of bacteria, when constrained by immigration, more prominent and easily quantifiable TAR patterns arise. The model fitting process supports this by estimating extremely low immigration rates for machine habitats. Despite this, machine habitats had the lowest mean \textit{A\textsubscript{crit}} (6.56 x 10\textsuperscript{3} cm\textsuperscript{2}). \textit{A\textsubscript{crit}} is not only affected by immigration as in my primary hypothesis, but can also be affected by number of niches (\textit{K}), density ($\rho$) and $\theta$. In the fitted model the low \textit{A\textsubscript{crit}} for machine habitats in spite of their isolation is caused by the low \textit{K} values of homogenous, man-made environments, a characteristic of these unusual habitats that warrants further investigation. \\
%DONE
\noindent The non-significance of habitat type, despite marked differences in the mean \textit{A\textsubscript{crit}} is due to the large, overlapping estimate ranges. Overall, after removing the outlying datapoint and removing habitat type as an explanatory variable, the model accounts for nearly half of the variation in log \textit{A\textsubscript{crit}} using the broad taxonomic groups.\\


%\noindent The results of this investigation support the theory that critical area will be larger for pathogens  and lower for bacteria (3.02 x 10\textsuperscript{3} cm\textsuperscript{2}), but protozoan critical area (7.43 x 10\textsuperscript{23} cm\textsuperscript{2}) was higher than algae (2.56 x 10\textsuperscript{19} cm\textsuperscript{2}) or fungi (1.72 x 10\textsuperscript{16} cm\textsuperscript{2}), in contrast to my prediction. 

%As habitat type was not significant in predicting log \textit{A[\textsubscript{crit}} I cannot address the hypothesis that more isolated habitats will have lower log \textit{A[\textsubscript{crit}}. Examination of the mean differences between habitats suggests that terrestrial habitats showed higher critical area (2.36 x 10\textsuperscript{30} cm\textsuperscript{2}) than lacustrine (1.28 x 10\textsuperscript{19} cm\textsuperscript{2}) and plant habitats (2.01 x 10\textsuperscript{4} cm\textsuperscript{2}) as predicted, but machine habitats had a lower critical area than expected (6.56 x 10\textsuperscript{3} cm\textsuperscript{2}). 

%Which datasets were successfully fit to the model and why? What do they have in common? Where do they differ in terms of habitat, taxonomic group and study design. 

%Which datasets weren't successfully fit to the model and why?


%Why did the riverine habitat and archaea taxa not have any successful fittings?
%DONE
\noindent The reason for the lack of successful fittings for riverine habitats is due to the low number of data points in each of these studies (see Table 7.1, Supplementary Material). Whilst the majority of these datasets had high R\textsuperscript{2} values, once adjusted R\textsuperscript{2} values were calculated the fittings were unsuccessful. Only one dataset included archaeal TARs and no significant relationship between area and OTU richness was found. This is likely due to the importance of environmental filtering for extremophile OTUs in soda lakes \cite{LanzenAnders2013SPaE}. It is interesting to note other datasets removed from the fitting process due to a lack of positive TARs. These included, fungi in the Antarctic cryoconite holes of two glaciers where extreme biomass influx negated observable TARs (datasets 10 \& 11) \cite{darcy2018island}. Inappropriate diversity metrics and spatial scaling may have led to undetectable toot-symbiotic fungi TARs (dataset 18) \cite{davison2018microbial}. Fungi OTU richness did not increase with area on submerged leaves due to a lack of energy increase with corresponding area as expressed by the species-energy theory (dataset 39) \cite{FeinsteinLarryM2012Tran}  \cite{wright1983species}. TARs may not have arisen in protozoan communities in submerged substrate due to a failure to reach equilibrium (dataset 55) \cite{henebry1980effect}. It is clear that the factors affecting microbial TARs are diverse and each habitat/taxa pairing may require unique assessment. \\ 

%DONE
%What was the anomalous result? Why was it anomalous?
\noindent The anomalous result removed from analysis concerned pathogenic bacterial OTU richness on 'true' geographic islands \cite{jean2016equilibrium}. The model was a poor fit to the data (R\textsuperscript{2}=0.23, adjusted R\textsuperscript{2}=0.18) and it's likely the error associated with estimating density for pathogenic bacteria over such large geographic scales lead to poor estimations of the remaining parameters and an excessive critical area estimate.   \\

%DONE
%What are the caveats that apply to this study? (Leave out caveats that apply to all studies.)
\noindent Whilst the data here indicate that habitat type is non-significant in predicting log \textit{A\textsubscript{crit}}, the large variation in mean log \textit{A\textsubscript{crit}} suggests there many be too few data points to discern a significant pattern. I also encountered challenges when trying to compare studies that used a variety of methods and quantification techniques. Microbial OTUs inhabit three-dimensional habitats and whilst steps have been taken here to account for this there is more work to be done to incorporating this fully. In a future extension of this mode I would consider each stratification of a habitat separately, to account for spatial heterogeneity. Volume has been shown to be more accurate in quantifying microbial TARs \cite{van2006bacterial}. It would be useful to further modify the model to explicitly incorporate volume and \textit{V\textsubscript{crit}} across datasets, as nearly all of them concern habitats within a volume even though often only surface area data is provided. Here I have used area with a depth metric (Depth Model) which suggests the habitat maintains the same area for the full depth, whereas natural habitats rarely take this shape and this reduces the accuracy of my results. \\

%DONE
\noindent Another issue I encountered was density estimations. The model required an estimation of individual density per unit area, however, direct counts are rarely given for microbial OTUs. Estimations were made in various ways, using gene sequence numbers or proxy papers, although these methods introduce error into the model fitting process. It would be beneficial to this project to develop more robust methods for estimating density as data taken from proxy papers introduces error into the fitting process. A broad scale experiment to quantify microbial TARs in a laboratory, where data specific to the needs of these model could be collected (i.e. density), could provide a more vigorous assessment of the models applicability to microbial TARs. \\

%DONE
\noindent When validating my model fitting procedures, error between true and estimated results increased with increasing parameter values. As I increased simulation parameters in parallel with each other (e.g. an increase in $\theta$,  was coupled by an increase in \textit{m\textsubscript{0}} and \textit{K}), the source of the increasing error is difficult to discern. Parameter ranges of speciation rate, \textit{m\textsubscript{0}} and \textit{k} given to the simulations were not inferred from microbial ecological theory, but were selected for ease of computation. A more thorough exploration of the parameter space, with ecologically relevant parameter ranges, to further validate the model fitting procedure and the areas of parameter space where there may be errors in fitting would be desirable in further work. \\

%What might be done about them? (Very important in a project write-up -- What would you do differently if you were doing the project again or had more time?)

%What future work could build more broadly on what we’ve found?
%DONE
\noindent There remains to be a thorough synthesis between biogeography and microbial ecology. Here I have gone some way to evaluate the influence of immigration on microbial TARs, however more work is needed to examine dispersal barriers. Dormancy is a widespread microbial response that may allow OTUs to overcome dispersal barriers and increase immigration to new habitats. However, it is a slow, passive process that will not necessarily lead the individual to a viable habitat \cite{LoceyKennethJ2010Stbw}. To fully elucidate the interplay of microbial ecology and biogeographic patterns, work is needed to incorporate dormancy as a biogeographic response.  \\         

%What are the implications of this work?
%DONE
\noindent An implication of this work is that if we can identify the niches within a habitat and the taxonomic groups that tend to occupy those niches, we may be able to better predict OTU richness at a range of spatial scales. This presents a complex challenge that requires the integration of environmental variables and habitat stratification. If these challenges could be overcome, it would be a particularly useful tool in predicting colonisation of new habitats such as soil exposed by glacier retreat, thus helping us model the biogeochemical processes this colonisation will produce. \\      

%A nice wrap-up, emphasising how this study in this system is of interest to people who work on other things, or other systems.
\noindent This study has demonstrated that microbial communities in isolated 'island' habitats can be subject to deterministic biogeographic mechanisms such as niche-structuring, before a critical area of transition (\textit{A\textsubscript{crit}}), to stochastic mechanisms of colonisation and extinction. I have also shown that taxonomic group is significant in predicting \textit{A\textsubscript{crit}}, but habitat type is not. The overwhelming number and complexity of microbial life, as well as the vital role these organisms play in ecosystem functioning, illustrates the importance of elucidating their biogeographic patterns. I hope that my study will lead to further research into the presence of deterministic and stochastic mechanisms in microbial biogeography, as well as the importance of taxonomic group on the relative influence of these processes. The synthesis of microbial ecology and biogeography will be of increasing interest as climate change alters habitats, creating and removing barriers, extending the range of pathogenic OTUs and leading to climate feedback loops of mineral and nutrient cycling. Microbial biogeography is an essential area of study in our global challenge to predict and mitigate the impacts of climate change. Everything is \textit{not} everywhere, and everything is changing.             