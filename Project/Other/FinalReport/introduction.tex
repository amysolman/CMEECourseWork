\chapter{Introduction}

MacArthur and Wilson's theory of island biogeography \cite{MacArthurRobertH1967Ttoi} is widely accepted as a fundamental ecological law. It posits that island communities are maintained by a combination of immigration and niche availability. Niches provide reduced inter- and intraspecific competition, slowing competitive exclusion. Immigration provides new species and new individuals, offsetting loses to competitive exclusion and genetic drift. Islands with a larger area and reduced distance from mainland communities or other islands, will exhibit higher species diversity at colonization-extinction dynamic equilibrium than those smaller and further away. A great deal of empirical evidence has been amassed to support the theory of island biogeography and it has been widely applied in understanding the effects of habitat fragmentation \cite{haila2002conceptual}. \

\indent Despite the popularity of this theory, there is empirical evidence to suggest it cannot always be applied to smaller islands \cite{triantis2006re}\cite{sfenthourakis2009habitat}. MacArthur and Wilson noted that archipelagos showed unusual SARs, with smaller island species-richness varying independent of size \cite{MacArthurRobertH1967Ttoi}. This exception to MacArthur and Wilson's putative ecological law has been dubbed the small island effect (SIE). Several hypotheses have been offered to explain the SIE. The 'subsidized island biogeography' (SIB) hypothesis suggests that smaller islands have a greater edge:interior ratio, thus receive a greater amount of nutrients per unit area in contrast to larger islands \cite{barrett2003small}\cite{anderson2001subsidized}. Moreover, extinction rates on islands may operate independently of area due to their environmental instability and high temporal turnover, where major episodic disturbances periodically wipe out colonizing species \cite{chisholm2016maintenance}. Thirdly, the 'habitat hypothesis' suggests a limited suite of habitats on small islands, in contrast to larger islands, limits species diversity \cite{triantis2008evolutionary}. \

\indent Chisholm \textit{et al} \cite{chisholm2016maintenance} have developed a unified theory to explain this biphasic island SAR. They posit that this pattern of species-richness is due to a transition from a niche-structured regime on smaller islands, to a colonization-extinction regime on larger islands. The niche-structured regime is characteristic of deterministic niche theories where environmental filtering, biotic interactions and interspecific trade-offs determine species richness \cite{chase2011disentangling}. The colonization-extinction regime is characteristic of stochastic theories such as the theory of island biogeography and neutral theory, where richness is dictated by colonization and extinction rate as well as ecological drift \cite{hubbell2001unified}.\

\section{Microbial Communities}

\indent Whilst numerous studies have looked at macro-organism SARs, relatively little is known about how microbial species are affected by habitat size, isolation and niche availability. Understanding the factors that regulate microbial species richness and community structure is important as they play a significant role in ecosystem functioning \cite{griffiths2011bacterial}. Greater understanding of the patterns underlying microbial biogeography is also important in predicting the responses of these organisms to a changing environment \cite{bradley2017microbial}.\

\indent Debate around the applicability of SARs to microbial systems stems from the idea that they are ubiquitous in the environment and functionally redundant. This was famously articulated by Bass-Becking \cite{baas1934geobiologie} who wrote, 'everything is everywhere, but, the environment selects'. Some studies have shown that SAR theories are not useful in describing microbial richness patterns \cite{fierer2006diversity} \cite{henebry1980effect}. \

\indent Investigating microbial biogeography also has its practical challenges. Problems often arise in experimentally manipulating habitats to a degree useful for such a study and it is only recently that advances in molecular tools has lead to a resurgence in the field. One study found bacterial SARs in aquatic treehole habitats comparable with that of larger organisms \cite{bell2005larger}. Organic aggregates have also been used as 'islands' to explore the biogeography of aquatic pathogens \cite{lyons2010theory}, showing a weak positive correlation between ‘island’ size, community metabolic response and functional diversity. Investigation of phytoplankton SAR in water bodies found evidence for the SIE, however they also detected a large lake effect where species richness declined beyond a threshold area as wind-induced mixing increased habitat homogeneity \cite{varbiro2017functional}. This indicated that small-scale niche relations were the most important determinants of species richness at the smallest spatial scales. The SIE has also been seen in benthic diatoms sampled from ponds on a wide spatial scale \cite{bolgovics2016species}. Increased niche dimensionality has been shown to increased functional diversity, with strong evidence for niche filtering of microbial taxa communities \cite{KevinLee2016Nfob}.\

\indent The majority of studies looking at microscopic island biogeography have focused on aquatic communities. This is likely due to the availability of isolated water bodies and their range of spatial scales. Bacteria also represent a major contributor to soil biodiversity and processes \cite{griffiths2011bacterial}. Despite this, little is known about belowground regulators of biodiversity. An in-situ study of ectomycorrhizal fungi communities within 'tree island' root systems showed that total species richness increased significantly with island size, but distance had little effect \cite{peay2007strong}. Despite the strong SAR found in this study, there was no good evidence linking the SAR to increases in niche variety with habitat size, a result consistent with the theory of island biogeography and neutral community models. Most investigations have manipulated habitat area only, inferring that niche availability will naturally vary with habitat size. One study directly manipulated niche dimensionality by varying resource richness and found this resulted in increased functional dissimilarity, community productivity and reduced invasion \cite{eisenhauer2013niche}. \

\indent To tackle the limitations of previous in-situ research where environmental variables were difficult to control \cite{fenchel2005bacteria}, this study aims to quantify the contributions of immigration and niches to diversity in a soil microbial community, by manipulating mesocosm 'island' habitats. We aim to assess whether Chisholm's unified theory of biphasic island SARs is supported by our results. The parsimonious machanistic model used by Chisholm \textit{et al} \cite{chisholm2016maintenance} to approximate the processes of a biphasic SAR is fit to the data, along with other mathematical models. The number of microbial species maintained by niches is inferred from the asymptotic richness as immigration rates are reduced. The contribution of immigration is inferred from the rate of richness increase across treatments. This investigation seeks to identify nonlinearities in the richness versus immigration curve. \

\section{Hypotheses}

\indent It is hypothesise that: \ 

1) Soil microbial communities will be dictated by a niche-structure regime at the smallest spatial scales, before transitioning to a colonization-extinction regime at larger spatial scales. \

2) 'Islands' with lower rates of immigration will have less species at the colonization-extinction dynamic equilibrium. \
