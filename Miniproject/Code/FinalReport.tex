\documentclass{article}
\usepackage[utf8]{inputenc}
\usepackage{setspace}
\usepackage{lineno}
\usepackage[margin=0.5in]{geometry}
\linenumbers
\doublespacing

\title{CMEE Miniproject: Population Growth \\ 
\large How well do different mathematical models, e.g., based upon population growth (mechanistic) theory vs. phenomenological ones, fit to functional responses data across species?}
\author{Amy Bo Tinky Solman\\
Imperial College London}
\date{24th January 2020}

\begin{document}

\maketitle

\section{Introduction}

The study of population growth, or dynamics, began in earnest in the early 20th century as a way to manage and predict agricultural stocks in the wake of the World Wars \cite{KingslandSharonE1995Mn:e}. Since that time the field has developed and been applied to conservation efforts and in predicting the impacts of climate change \cite{ozgul2010coupled} \cite{hunter2010climate}. Fluctuations in population abundance can impact keystone processes and ecosystem dynamics \cite{sinclair2003mammal}. Emergent functional characteristics such as disease transmission can also be affected by host/pathogen population dynamics \cite{alexander1996population}. Moreover, population dynamics for species such as phytoplankton dictates rates of carbon fixation in aquatic systems \cite{reynolds2000regulation}.\ 

\indent Turchin asserts that there are three general laws of population dynamics: populations tend to grow exponentially, population growth is self-limiting and consumer-resource populations tend to oscillate \cite{TurchinPeter2001Dpeh}. Patterns of population dynamics include exponential growth, logistic growth and predator-prey interactions \cite{RockwoodLarryL.2015ItPE}. Exponential population growth describes a continually increasing number of individuals under conditions where resources are not limiting \cite{BotsfordLouisW.author2019Pdfc}. Logistic growth occurs when population growth rate decreases as it reaches carrying capacity \cite{KingslandSharon1982TRMT}. Predator-prey interactions are concerned with the feeding relationship between species, resulting in a temporally staggered undulating pattern, with predator numbers lagging behind prey \cite{BerrymanAlanA.AlanAndrew1937-2008Ps:a}.  \

\indent  Population ecologists are concerned with quantifying these patterns of growth. Levin's stated that, "it is of course desirable to work with manageable models that maximise generality, realism and precision" \cite{levins1966strategy}. Over the past century a wealth of models have been devised to fit the various patterns of population growth observed in nature. Models may be either phenomenological or mechanistic. Mechanistic models often need a large number of parameters to describe complex ecological systems, whilst phenomenological models may need relatively few \cite{transtrum2016bridging}. In mechanistic models then we loose generality, in gaining precision and realism. In phenomenological models we may loose sight of the real-world mechanisms at play.\

\indent Understanding microbial patterns of population growth is important for food microbiology, risk assessment and water protection \cite{zwietering1990modeling} \cite{nauta2002modelling} \cite{dukan1996dynamic}. Microbial growth patterns also play a significant role in ecosystem processes, particularly in habitats newly exposed through climate change \cite{bradley2017microbial}. Typically, bacteria in culture has a four phase growth curve: the lag phase, exponential or log phase, stationary and death phase \cite{al2008studying}. Two types of models can be applied to bacterial growth: 1) models concerning changes in population over time; 2) models predicting changes in population in changing environmental conditions; 3) models combining the two \cite{whiting1995microbial}. \

\indent  In this investigation I compare a variety of both mechanistic and phenomenological mathematical models, to see how well each characterises patterns of bacterial growth over time. Here I use a variety of statistical measures to find which model best fits the bacterial sigmodial growth curve see in the given dataset.

\section{Methods}

\subsection{Computing Tools}

\section{Data}

\section{Results}

\section{Discussion}


\bibliographystyle{plain}
\bibliography{bibliography}

\end{document}
