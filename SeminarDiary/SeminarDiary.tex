\documentclass{article}
\usepackage{graphicx}
\usepackage[margin=1in]{geometry}
\usepackage{subcaption}
\usepackage{csvsimple}
\usepackage{booktabs}
\usepackage{pgfplotstable, colortbl} %for making tables from csv files
%\pgfplotset{compat=1.16}
\usepackage{longtable} %to fix my long table
\usepackage{tabu} % if you want - to fix a wide table
\usepackage{textcomp} % for degree symbol

\begin{document}

\title{Seminar Diary{}}
\author{Amy Solman}

\maketitle

\section{Deep-time evolution of biological responses to temperature changes 10th October 2019 }

Dimitrios Georgios Kontopoulos - Imperial College London \bigskip

\noindent To better forecast the impacts of climate change we need to look at how biological traits are thermodynamically constrained. The measure of biological traits at different temperatures typically produces a unimodal thermal performance curve (TPC).  The literature details multiple TPC hypotheses. These range from strong thermal dynamic constraints (‘Hotter-is-better’) to weak thermal dynamic constraints (‘Perfect biochemical adaptation’). \bigskip

\noindent Dimitrios Georgios Kontopoulos tested two hypotheses regarding the evolution of thermosensitivity: that thermal sensitivity evolves around an optimum value or evolves in other ways. To test these hypotheses, Kontopoulos used a phylogenetic comparative approach, looking at the correlation structure of parameters and their phylogenetic signal. The phylogeny showed random bursts of trait evolution. There appeared to be no global optimum, instead species explore the parameter space through evolutionary time. The effect of temperature on mutation was also explored. Simulations showed that mutations became more destabilising with temperature. Further simulations with multiple species at multiple temperatures indicated weak evidence that higher temperatures reduce mutation rate. Kontopoulos posits that a multidisciplinary, multi-level approach is needed to develop a unified picture of thermal adaptation. We need to look at varying levels of organisation, from genome composition to how species interact.

\section{A manifesto for systematically describing consumer-resource interactions 31st October 2019}

Daniel Barrios-O’Neill - Leverhulme Trust, University of Exeter  \bigskip

\noindent Consumer-resource interactions are at the heart of ecology and worth quantifying. Capture rate contains information about the space in which the interaction occurs, whereas handling time relates to organismal biomechanics. When quantifying encounter rates, the habitat physical structure and organismal biomass must be considered. Physical structure is important due to the rate of anthropogenically driven change in natural environments. Biomass is important because larger, warmer animals have higher capture rates and shorter handling times. Mutual interference suggests that as the density of predators in a patch increases, per capita consumption rate decreases. It has also been shown that consumers moving through volume encounter each other more frequently than surface foragers, leading to a steeper scaling relationship of mutual interference.\bigskip

\noindent In consumer-resource interactions, metabolic predictors are well established. However, the prevalence of model species skews the global data. To understand these interactions, modifiers of encounter rates are key and knowledge gaps must be addressed. Experiments exploring consumer-resource interactions must consistently measure all appropriate variables, so that the data can be used by others. Opportunities to harvest that data are myriad, especially when combined with open access to data and tools. The large global ecological community must work together, and treat undescribed interactions with the same reverence as undescribed species. 

\section{Flowers, bees and shifting seasons – how to adapt when Nature’s calendar goes out of sync in a warming world 21st November 2019}

Jacob Johansson - Theoretical Population Ecology and Evolution Group (The PEG), Lund University, Imperial College London \bigskip
  
\noindent In recent decades there have been large shifts in biological events, including; flowering times, butterfly emergence and bird migrations. There is variation in the rate of change among species and events. A major concern is that phenological mismatch may affect community function. Mismatch may have negative fitness consequences, as adaptive responses track seasonal optima. Demographic consequences are comparatively understudied. Johansson's literature review indicated mixed demographic responses to climate change. \bigskip

\noindent Plants and bees both exhibit an initial growth phase, followed by a switch to a reproduction stage. As production increases, optimal switching time moves closer to the end of the season. Optimal switching time may be dependent on variation in seasonal production rate, as well as total growth capacity. Size dependent relative growth rates in plants and bees have shown that those species without exponential growth should reproduce earlier to increase productivity. As season length increases, reproducing earlier would avoid competition, whilst switching later would lead to a larger population size. Different species may experience asymmetric shifts, thus resulting in changes to interspecific resource competition. \bigskip

\noindent Demographic responses to climate change may show short term declines, but long-term population increases due to competitive release. Adaptive responses may include evolutionary rescue to restore population sizes. Overall, the effects of climate change on community function will be varied and dependent on the unique features and adaptations of each system.   

\section{Managing fisheries to protect dependent predators 16th January 2020}

Simeon Hill - British Antarctic Survey \bigskip

\noindent Ecosystem-based management recognises that the fish species we exploit are embedded in complex ecosystems. Its goals are to maintain ecosystem productivity, health, resilience and services. To action this, we need to define safe ecological limits. The Convention on the Conservation of Antarctic Marine Living Resources (CCAMLR) defines safe ecological limits as preventing changes to the ecosystem that are not reversible within two to three decades. Strategic ambiguity is often used to reduce the level of detail in defining safe ecological limits and may foster conflicting opinions on management strategies between stakeholders.  \bigskip

\noindent Traditionally, fisheries management has used a single species management approach. The aims are to maintain the harvested population and achieve socioeconomic goals. Single species management relies on feedback loops derived from control theory. These loops are comprised of a variable (size of the fished population), reference points including targets (states to aim for), limits (boundary of states to avoid) and soft limits (boundary of states to avoid most of the time), as well as adjustment and implementation methods. A feedback loop can also be extended to include predators of the fished organism. Many fisheries fail to use reference points due to strategic ambiguity. \bigskip

\noindent Fisheries are not the only factor modifying fish/predator populations. Climate change and other anthropogenic drivers play important, yet equivocal roles. Precautionary measures may be used to limit fishery impacts on predators, albeit an ambiguous level of protection. Suitable reference points can also assist in protecting populations, but organism life-stage and data lag-time must be considered.  

\section{How shrinking glacier are affecting Alaska’s coastal ecosystems 20th February 2018}

Eran Hood - UAF Geophysical Institute Presents Science for Alaska Lecture Series
University of Alaska Southeast \bigskip

\noindent 90\% of Alaskan coastal glaciers are shrinking in mass, contributing to global sea level rise and impacting downstream ecosystems. Glaciers provide a number of ecosystem services, including food, fisheries, tourism and recreation, as well as regulating water quality, outburst floods and hazards. Rivers act as conduits between glaciers, ice fields and marine ecosystems. As these landscapes change, it is important to understand how rivers receiving runoff from glaciers contrast to those fed by forests. Forested streams receive stochastic discharges, while glacier fed streams receive deterministic discharge, in line with seasonal temperature change. Glacierized streams have consistently low temperatures and are less effected by air temperature. They also exhibit greater turbidity. \bigskip
 
\noindent Glaciers provide one of three stream types found in Southeast Alaska, along with clearwater and brownwater. The physical removal of glaciers allows for new streams to develop. Whilst most salmon return to their natal stream, some populate new streams, exploiting the habitat mosaic of varying resource availability. Glaciated areas releasing cold, turbid water, are hotspots for other ice associated species. Tidewater glacier fjords supply nutrients, increasing productivity and providing fertile hunting grounds for plunge feeding seabirds. Tidewater glaciers also produce ice bergs used by harbour seals as safe haul-outs and areas for pupping. \bigskip

\noindent Climate driven changes to ice fields have far reaching impacts. In order to predict and protect against these changes we need coordinated, interdisciplinary systems thinking.

\section{Arctic Microbes: Living in a Frozen Ocean 10th March 2011}

Marcela Ewert - University of Washington’s School of Oceanography and Astrobiology \bigskip

\noindent The Arctic ocean is a unique habitat, covered in saline ice and surrounded by land. A cubic meter of saline ice has between 10,000 and 100,000,000,000 pores and a surface area between 100,000 and 1,000,000 square metres. The saline pores filled with liquid water providing habitats for microorganisms. These microorganisms are major contributors to essential biogeochemical processes. \bigskip

\noindent One of the most abundant microorganisms to inhabit sea ice are algae. Single celled algae form long chains and are an important source of food, supporting at least half of Arctic ecosystems. Bacteria also help to recycle nutrients and maintain healthy systems, thus supporting larger Arctic species. These bacteria are halophilic and tolerate salt levels up to seven times higher than surrounding sea water. They also produce a protective substance around the cell wall to prevent penetration and damage from the volatile ice environment. \bigskip

\noindent The Arctic and Antarctic ice caps cover between 3\% and 6\% of the Earth’s surface. This ice serves as an interface between the ocean and the atmosphere. The Arctic ocean is part of our integrated planetary system and experiencing warming rates twice the global average. It is important to understand the microbial processes occurring within and around Arctic ice and how that is changing, in order to develop accurate climate models and strategies for the future.  
  
\section{The Secret Language of Bacteria: An ASM “Microbes After Hours” Event 28th January 2013}  

Bonnie Bassler - Princeton University \bigskip

\noindent Bacteria have existed on Earth for 4 billion years and can be both virulent and beneficial to other organisms. They carry out these processes by talking to each other with a chemical language. They recognise when they are among others of their species, they can count themselves and carry out behaviours as a multicellular group. \bigskip

\noindent Bacteria exist in two modes: social and a-social. A single cell detects if it is alone by producing small, hormone-like molecules called autoinducers. These are released into the environment and if they are not returned by sibling cells, a-social traits are expressed. As the cell divides, sibling cells release their own autoinducers.  Once a certain threshold is met these molecules can be detected by a receptor protein within the cell membrane and the population alters gene expression in unison. This is known as quorum sensing. Interspecific differences in autoinducer molecules infers intra-specific communication properties. \bigskip

\noindent Bacteria are also capable of multilingual quorum sensing. A second enzyme produces a universal communication molecule allowing for interspecific communication. Whilst bacteria use hormonal sensing to detect whether they are alone or in a group, they also measure the ratio between the ‘self’ and differential molecules to infer if they are the most prominent bacteria in a given environment. Understanding quorum sensing may lead to new approaches in antibiotics. In order to be successfully virulent, bacteria must act as a group. By modifying the shape of autoinducers we may be able to block bacterial receptors, inhibiting group behaviour. Behaviour modification could also be used to increase the impact of autoinducers, leading to beneficial industrial applications. Studying the quorum sensing process in bacteria can help us understand the evolution and robustness of multicellularity in higher organisms, as well as tackling some of humsanities most pressing medical and industrial issues. 

\section{Astrobiology and Space Exploration Introduction 27th August 2009}

Seth Shostak - Senior Astronomer at the SETI Institute \bigskip

\noindent Throughout history, humans have held varying beliefs about the Solar System and what lies beyond. Central to refining these ideas has been the ability to measure the distances of celestial bodies, allowing us to look back into the past and speculate on the future. In 1823 Heinrich Olber described Olber’s paradox, whereby the darkness of the night’s sky was in conflict with an infinite and fixed universe. The stars should constantly illuminate the point of view of the observer. The paradox allowed astronomers to hypothesise that all matter in the universe must be moving away, causing the light of distant stars to appear redder and dimmer. In the early 20th century, astronomer Henrietta Leavitt discovered that the brightness of a nebula could be calculated by measuring the length of its luminosity cycle. Edwin Hubble further developed the work of Leavitt and used stars with known distances and luminosities, known as Cephid Variables, to estimate the distance of extra-galactic nebula and assert that they were other galaxies moving away from us. \bigskip

\noindent By working the expanding universe backwards, astronomers have been able to suggest that there was a time when all matter was tightly condensed. This theory became mockingly known as the Big Bang. Despite the success of this theory, there remain many questions: what existed before the Big Bang? What did the universe expand into? \bigskip

\noindent There are several theories concerning the future of the universe. By looking at supernovas, astronomers have calculated that the expansion of the universe is speeding up due to dark energy. It may be that dark energy continues to expand space until all matter is pulled apart. There may come a time when the universe begins to re-condense. Ultimately, after a finite period of habitability, it seems the universe will become an infinity of nothingness.

\section{Astrobiology and the Very Small 1st April 2014}

Ken Kubo - American River College, Sacramento \bigskip

\noindent Astrobiology seeks to understand the origins of life on Earth and its existence elsewhere. Defining life can be challenging as many non-living things share living properties. NASA scientist Gerald Joyce defines life as, “a self-sustaining chemical system capable of undergoing Darwinian evolution”. \bigskip

\noindent Habitability is an important concept in astrobiology, informing policy and mission planning. The NASA Astrobiology Roadmap defines habitability as an environment with, ‘extended regions of liquid water, conditions favourable for the assembly of complex organic molecules, and energy sources to sustain metabolism’. Water is key to habitability due to the low density of ice insulating aquatic ecosystems that would otherwise freeze. It is also abundant and an excellent solvent. \bigskip

\noindent Other planets in our Solar System present extreme habitats. Study of terrestrial extremophiles can help us understand the lifeforms that may exist elsewhere. Microorganisms, such as \textit{Methanopyrus kandleri}, found living around oceanic black smokers, can survive in temperatures up to 121\textdegree C. Microorganisms also thrive in extreme cold environments. Lake Vostok, Antarctica, is buried under ice, with temperatures reaching minus 13\textdegree C and contains upwards of 4000 species. \textit{Deinococcus radiodurans} thrives in highly radiated environments, surviving exposure to 1.5 million rads. \bigskip

\noindent Astrobiology research has focused on Mars and Jupiter’s moon, Europa, as potential habitats for life. Images of the Martian Gale Crater suggest the area was once a freshwater lake, providing conditions favourable to life. The Martian Meteorite Yamato found in Antarctica, indicated there had once been liquid water on Mars and it may have been exposed to biological processes. Europa’s subsurface ocean is warmed by tidal heating and may host aquatic life. \bigskip

\noindent Astrobiology provides a cosmic context in which to understand the origins of life on this planet, its future, and the possibility of life on other worlds.  

\section{Effects of Temperature on Microbial Metabolic Rates: Linking Individual Responses to Ecosystem Impacts 23rd January 2020}

Tom Smith - Imperial College London \bigskip

\noindent Microbes play an essential role in biogeochemical cycling. In order to construct accurate climate models, we need to understand the impacts of temperature change on microbial metabolic rates. Tom Smith's analysis of a large range of macroorganism thermal performance curves (TPCs) revealed a global average thermal sensitivity (E) of 0.65. Species follow similar patterns of inter- and intra- specific E. Do microorganisms conform to the same rules? Smith et. al used digitisation software to extract 542 microbial TPC datasets from the literature. E for bacteria and archaea was found to be greater than the macroorganism global average. Environmental data supports these findings, as habitats with higher proportions of microbes have higher E overall. \bigskip   

\noindent How do these short-term responses link with long-term responses? Generally bacterial species show increased metabolic rate with temperature. We may then expect climate change to increase microbial metabolic rates. Some species may experience a decline in metabolic rate, unless they can adapt to higher temperatures. Lab settings have shown rapid adaptation to temperature change, but can this be replicated in natural environments? Genetic and phenotypic variation within microbial communities may facilitate species sorting, as an alternative to adaptation. Evidence supports species sorting over adaptation, with changes in optimum temperature phylogenetically dependent. Climate change will likely cause a shift in microbial community thermal optima through either or both of these processes. Understanding the extent to which climate change may increase ecosystem respiration depends on the relative abundances of autotrophs to heterotrophs, and eukaryotes to prokaryotes. Global estimations of these proportions vary greatly. Future work could benefit from empirical data collection through mesocosm manipulation.      

\section{Coronaviruses 20th February 2020}

Michael Tristem - Imperial College London \bigskip

\noindent The coronavirus group possess +ssRNA genomes, similar to polio, and envelopes similar to influenza. Nucleoprotein forms a helical shape around RNA in the viral core. The group exhibit unusually complex genomes, larger than other RNA viruses. Within the Coronavinae subfamily, the beta virus genera are of the most clinical importance to humans. Beta human coronaviruses are common, accounting for a quarter of common colds, and are rarely pathogenic. SARS-CoV (Severe Acute Respiratory Syndrome Coronavirus) and MERS-CoV (Middle East Respiratory Syndrome Coronavirus) represent recent pathogenic beta human coronavirus outbreaks. SARS-CoV 2 and SARS-CoV are closely related, with only 380 amino acid differences. ACE2 (Angiotensin-converting Enzyme 2) is the cellular receptor utilised by SARS-CoV, and likely SARS-CoV 2. The distribution of ACE2 in the host body determines the maximum tissue tropism that infection can cause. ACE2 is generally found in the lungs, kidneys, heart and gastrointestinal tract of humans, thus infection can lead to multiple organ failure. \bigskip

\noindent It is believed the initial SARS-CoV outbreak was transmitted from bats via civet intermediaries, and the most recent SARS-CoV 2 transmitted via pangolin intermediaries in the Wuhan wet market (China), December 2019. Initial stages of the virus are similar to many other conditions and must be confirmed with genetic testing. Many are asymptomatic and at low risk of transmitting adequate viral load to infect others. Viral particles travel in oral droplets that can be absorbed through the eyes, inhaled or ingested. Survival of viral particles on materials varies, but transmission is most likely from hard surfaces. This unique coronavirus strain has proved a greater risk to human health than its predecessor. Further research is required to prevent future outbreaks of coronavirus strains.   


\end{document}